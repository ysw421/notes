\documentclass[../note.tex]{subfiles}

\providecommand{\mainfile}{standalone}

\ifx\mainfile\empty\else
    \usepackage{import}
    \title{Chapter 1 - Types of Memory and Brain Regions of Interest}
    \author{BCSC}
    \date{\today}
\fi

\begin{document}
\ifx\mainfile\empty\else
    \maketitle
    \newpage
\fi


\chapter{Types of Memory and\\Brain Regions of Interest}

\section{Cognitive Neuroscience}

\subsection{인지 심리학\textsuperscript{Cognitive psychology}\index{인지 심리학}}
사람의 정신 과정(e.g., 인식, 집중, 언어, 기억, 결정 만들기 등)을 연구하는 분야

\subsection{행동 신경과학\textsuperscript{Behavioral neuroscience}\index{행동 신경과학}}
동물로 부터 뇌의 메카니즘을 찾는 학문\\

\paragraph{Method: }
침습적 방법을 사용하여 동물 뇌의 메카니즘을 찾음 $\Rightarrow$ 궁극적 목표! 동물 뇌에 기반하여 사람 뇌의 메카니즘 탐구

\begin{figure}[h]
  \centering
  \includegraphics[width=0.5\textwidth]{image/cognitive_relationship}
  \caption{The relationships between the fields of cognitive psychology, cognitive
neuroscience, and behavioral neuroscience.}
  \label{fig:cognitive_relationship}
\end{figure}

\subsection{인지 신경과학\textsuperscript{Cognitive neuroscience}\index{인지 신경과학}}
인지 심리학과 행동 신경과학을 망라하는 분야로, 사람의 정신 과정과 뇌의 메카니즘을 탐구하는 학문, 참고: Figure \ref{fig:cognitive_relationship}


\section{Memory Types}
Learning and memory - Volume 1의 chapter 1.0.2 A Typology of Memory Terms(1주차 시원)와 내용이 유사함.
Learning and memory가 더욱 내용이 자세한데 반하여, 본 책은 tree로 각 기억의 상관관계를 명백히 명시함.

\begin{figure}[h]
  \centering
  \includegraphics[width=0.7\textwidth]{image/memory_tree}
  \caption{Organization of memory types.}
  \label{fig:memory_tree}
\end{figure}

Figure \ref{fig:memory_tree}는 본 절서 설명하는 모든 기억을 tree 형태로 나타낸 그림임.
memory는 단계별로 pair로 나타나며 explicit memory와 implicit memory로 나뉜 것이 첫번째 pair임.

\subsection{explicit memory \& implicit memory\index{explicit memory}\index{implicit memory}}
explicit memory와 implicit memory는 각각을 conscious memory와 nonconscious memory라고도 함.\footnote{
  learning and memory에서는 각각 용어는 `기억하는 행위에 초점'을 두거나 `기억한다는 것을 의식, 인지하고 있는가에 초점'을 두는지에 따라 구분하였음.
}
이전 경험에 대하여 의식적 경험/인식이 있으면 explicit memory, 아니면 implicit memory.

\begin{example}[explicit memory \& implicit memory]각각의 예시와 실험 방법(leanging and memory)\\
  \begin{itemize}
    \item explicit - 의식적으로 기억함(기억하는 행위에 초점)\\
	: e.g., 파리 여행 떠올리기, 암기 리스트의 단어 떠올리기\\
	: explicit memory test: 과거 기억줘!(요청) $\rightarrow$ (응답)
    \item implicit\\
  : implicit memory test: 작업 수행 요청 $\rightarrow$ 이전 작업이 어떤 영향 미치는지 측정\\
  실험 예: `코끼리(elephant)가 포함된 단어장 외우세요' $\rightarrow$ task: 학습한 단어이면 선택하세요!\\
  $\;\;\;\;$: elephant 선택 $\rightarrow$ 명시적
	: 2번 집단, 같은 단어장 위우고 $\rightarrow$ `다음으로 시작하는 단어 떠올리세요': ele\_\_\_\_\\
  $\;\;\;\;$: elephant! $\rightarrow$ 암묵적 (학습 안했으면? (star)elegant(star))\\
  $\;\;\;\;$: priming: 무의식적으로 다시 떠오르는 현상\\
  $\;\;\;\;\;\;\;\;$$\rightarrow$ elephant 증가 비율, 40\%(학습 후) - 10\%(학습 전) = 30\%
  \end{itemize}
\end{example}

\subsubsection{implicit memory}
implicit memory의 2번째 pair은 skills와 repetition priming이 있음

\begin{itemize}
  \item{\textbf{skills}}\index{skills}
    기술을 한번 연마하면 무의식적으로 행위 가능. e.g., 한번 자전거 기술 습득시 페달 각도, 균형 등 고려하지 아니함

  \item{\textbf{repetition Priming}}\index{repetition Priming}
    repetition priming은 반복하여 더 효율적이게 되는 것. e.g., 테레비 광고 반복적으로 노출될 경우 쇼핑하다 광고 상품 볼 경우 생각나서 구매 확률 증진
\end{itemize}

\subsubsection{explicit memory}
explicit memory의 2번째 pair에는 long-term memory와 working memory가 있음.
working memory의 경우 short-term memory라고도 부름\footnote{learning and memory에서는 두 용어를 명백히 구분하여 설명함. short-term memory의 경우 `정보가 분류되어 의식에 도달한 후에도 정보를 유지하는 것'을 말하며, working memory는 `정보 조작하는 동안 짧게 정보를 저장하는 것'을 말함. working memory는 보다 복잡한 과정임.}

\begin{definition}[old-new recognition]
  ``old''와 ``new''를 구분하는 것. 옛된 기억에 대하여 ``old''라고 대답하는 것은 정확한 기억임. 예컨대, 새로운 pw와 이전 pw??
\end{definition}
long-term memory와 working memory는 delay 동안 정보를 기억하고 있는지 여부에 따라 구별 가능.

\begin{center}
  \small
  \begin{spacing}{0.8}
    \begin{BVerbatim}

            +-------+                +-------+                +-------+
item 1 ---> | delay |--- item 2 ---> | delay |--- item 3 ---> | delay |--- item 4 ...
            +-------+                +-------+                +-------+


      | long-term memory | working memory |
------+------------------+----------------+
delay | minutes to hours |    seconds     |
  N   |    many items    |   a few items  |
    \end{BVerbatim}
  \end{spacing}
\end{center}

\paragraph{long-term memory}
long-term memory 실험에서는 여러개의 item을 사용하는데, 각 item 별 delay 시간(몇분에서 몇시간)이 길어 피험자는 기억을 유지하지 못함.
\index{long-term memory}

\paragraph{working memory}
working memory 실험에서는 적은 수의 item을 사용하며, delay 시간(초 단위)이 짧아 피험자는 기억을 유지할 수 있음.\footnote{learning and memory서는 mechanism 중심적, 본 책에서는 실험 기법을 중심적으로 working memory를 설명하는 것 같음}
\index{working memory}

+ explicit memory는 종종 long-term memory만을 나타낸다고 함.

\subsection{episodic memory \& semantic memory\index{episodic memory}\index{semantic memory}}
long-term memory의 3번째 pair는 episodic memory와 semantic memory.\footnote{leaning and memory서 long-term memory는 본 책에 비하여 범주가 넓음. visual-spatial memory, imagery 등 다양한 종류를 설명하며, 심지어 skill 또한 long-term memory로 설명함.}

\begin{itemize}
  \item{\textbf{episodic memory}}
    과거 episode와 관련된 상세한 기억. what occurred, where it occurred, and when it occurred\\
    e.g., 마지막으로 부모님을 보았을 때를 떠올림
  \item{\textbf{semantic memory}}
    오랜 기간에 걸쳐 학습된 기억. e.g., 단어의 정의\\
    언어 처리와 관련
\end{itemize}

+ 본 책서 long-term memory는 sementic memory를 제외한 (tree에 존재하는) 모든 long-term memory를 지칭함.

\begin{note}
  learning and memory서...\\
  \begin{itemize}
    \item Episodic Memory\\
      - 사건을 기억하는 neural mechanisms나 인지 절차에 바탕해서 공간 / 시간과 함께하는 기억\\
      - autonoetic consciousness: 과거 경험을 의식하여 떠올리는 능력\\
      - 논쟁 거리: 다른 동물도 episodic memory를 할까? $\rightarrow$ episodic-like memory 용어 등장\\

      실험실에서 쓰는 측정 기법
      \begin{itemize}
        \item free recall: 순서 없이 자유롭게 기억해내기
        \item serial recall
        \item cued recall: 단서 주고 기억해내기
        \item recognition: 익숙한 것들을 골라내기
        \item source judgment: 시각? 청각? 등
        \item recency judgement: 어떤것이 최근?
        \item frequency judgment: 몇번 발생?
      \end{itemize}

    \item Semantic Memory\\
      세상에 대한 기본 지식, 물론 어마어마한 양의 정보\\
      e.g., 단어, 의미, 개념 등등, 프랑스 수도나 세계 2차 대전(organize knowledge)

      episodic memories에서 유래 가능\\
      : episodic memories(수업 수강) $\rightarrow$ 시간 경과(수업 시간, 장소 등 정보가 점점 사라짐), 반복 $\rightarrow$ semantic memory

      episodic과 semantic memory의 경계가 쉽게 모호해질 수 있음\\
      e.g., `최근에 수업에서 뭐 배웠니?' $\rightarrow$ 시간적인 면 + 배운거에 대한 단서

      word-priming technique: semantic memory를 연구하는 강력한 tool
        두 단어를 보이고 두번째가 진짜 단어인지 가짜 단어인지(word-nonword) 판단하게함\\
        - nurse $\rightarrow$ doctor 빠름\\
        - shoe $\rightarrow$ doctor 느림\\
        $\rightarrow$ 연관된 단어는 속도가 빠름. semantic memory는 network처럼 연결되어 있음을 알 수 있음
  \end{itemize}
\end{note}

\subsection{context memory \& item memory\index{context memory}\index{item memory}}
\label{subsec:contextmemoryitemmemory}
4번째 pair는 context memory와 item memory.

\begin{itemize}
  \item{\textbf{context memory}}
    item은 2가지 문맥(context) 중 하나로 제시됨.\\
    e.g., 보라 \& 민트, 왼쪽 \& 오른쪽\\
    old와 new를 인식(recognition)하는 것에 기반하ㅁ\\
    e.g., 사과를 본적 있음???????
  \item{\textbf{item memory}}
    item에서 문맥을 회상(recall)함\\
    e.g., 그 사과가 무슨 색이었었었지?
\end{itemize}

+ 문맥은 정보의 원천이므로 context memory를 source memory라고 부르기도 함.\index{source memory}

+ 2개의 item의 연관성에 대한 associative memory는 context memory와 유사함.\index{associative memory}

\subsection{remembering \& knowing\index{remembering}\index{knowing}}
5번째 pair는 remembering과 knowing.

\begin{itemize}
  \item{\textbf{remembering}}
    주관적인 경험을 상세히 떠올리는 것\\
    e.g., 차 어디에 주차하였지???????\\
    일반적으로 context memory와 관련
  \item{\textbf{knowing}}
    주관적인 경험을 상세하지 아니하게 떠올리는 것\\
    e.g., 저 인간 어디에서 본 적이 있어! 하지만 어디서였는지는 기억나지 아니해.\\
    일반적으로 item memory와 관련
\end{itemize}

\subsection{recollection \& familiarity\index{recollection}\index{familiarity}}
6번째 pair는 recollection과 familiarity.

\begin{itemize}
  \item{\textbf{recollection}}
    all the forms of detailed memory\\
    i.e., episodic memory, context memory, and ``remembering''\\
    describe strong memory
  \item{\textbf{familiarity}}
    all the forms of non-detailed memory\\
    i.e., semantic memory, item memory, and ``knowing''\\
    describe weak memory
\end{itemize}

\subsection{Tulving의 논문}
Endel Tulving는 remembergin과 knowing을 나눔. 증거:
\begin{enumerate}
  \item 과거 상세한 것을 기억 못하지만 단어를 아는 환자(i.e., remembering을 못하는 환자)
  \item introspection : 자신의 정신 과정을 들여다 보는 것, remembering과 knowing의 정신 과정이 다름.
    참고: Figure \ref{fig:tulving_experiment}
\end{enumerate}

\begin{figure}[h]
  \centering
  \includegraphics[width=0.5\textwidth]{image/tulving_experiment}
  \caption{Probability of“remember” or
“know” responses as a function of confidence
judgements (key at the top right). Generated using data from Tulving (1985)}
  \label{fig:tulving_experiment}
\end{figure}


\section{Brain Anatomy}
뇌의 구성 요소: 후두엽(occipital lobe), 측두엽(temporal lobe), 두정엽(parietal lobe), 전두엽(frontal lobe)

\tbox{gray matter \& white matter}
{
  \vspace{-2em}
  \begin{itemize}
    \item \textbf{회질}: 각 lobe의 표면, 세포체로 구성되어 있음\index{회질}
    \item \textbf{백질}: 각 lobe의 표면의 내부, 엑손으로 구성되어 있음\index{백질}
  \end{itemize}
}

\tbox{각 엽의 역할}
{
  \vspace{-2em}
  \begin{itemize}
    \item \textbf{후두엽(occipital lobe)}: 시각 처리\index{후두엽}
    \item \textbf{측두엽(temporal lobe)}: 시각 처리, 언어 처리\index{측두엽}
    \item \textbf{두정엽(parietal lobe)}: 시각 처리, 집중\index{두정엽}
    \item \textbf{전두엽(frontal lobe)}: 많은 인식 처리\index{전두엽}
  \end{itemize}
  뇌의 절반 이상이 시각 처리와 관련되어 있음 $\Rightarrow$ 기억 연구에서는 일반적으로 시각 항목(e.g., 단어, 사진)을 자극으로 사용
}

\tbox{cortex\index{cortex}}
{
  Figure \ref{fig:brain_views}는 기억과 관련된 뇌의 지역들을 보여줌.
  \begin{center}
    \includegraphics[width=0.7\textwidth]{image/brain_views}
    \captionof{figure}{Brain regions associated with memory. Each region is shown within red ovals
    and labeled.}
    \label{fig:brain_views}
  \end{center}

  \ttbox{gyrus \& sulcus\index{gyrus}\index{sulcus}}
  {
    \vspace{-2em}
    \begin{itemize}
      \item \textbf{gyrus(복수: gyri)}: 튀어나온 부분, Figrue \ref{fig:brain_views}A의 연한 회색 부분들
      \item \textbf{sulcus(복수: sulci)}: 홈, Figrue \ref{fig:brain_views}A의 어두운 회색 부분들
    \end{itemize}
    과학 기사에서 주로 뇌 활성화는 특정 gyrus와 sulcus에 국한되엉 있음
  }

  \ttbox{기억과 관련된 엽과 피질}
  {
    \vspace{-2em}
    \begin{itemize}
      \item \textbf{내측 측두엽(medial temporal lobe)}: 해마와 주변 피질로 구성\index{내측 측두엽}
      \item \textbf{배측 전전두엽 피질(dorsolateral prefrontal cortex)}: 운동 처리 영역 앞쪽의 dorsol과 lateral로 구성\index{배측 전전두엽 피질}
    \end{itemize}
  }

  Figure \ref{fig:gyri_sulci}는 기억과 관련된 gyri와 sulci를 보여줌. 다른쪽 반구 또한 구조적으로 동일함.
  \begin{center}
    \includegraphics[width=0.7\textwidth]{image/gyri_sulci}
    \captionof{figure}{Gyri and sulci in brain regions of interest. Left, lateral view of the left hemisphere
(occipital pole to the right). Right, inferior view of the left hemisphere (occipital pole at
the bottom).}
    \label{fig:gyri_sulci}
  \end{center}
  
  처음으로 시각 감각이 처리되는 영역은 V1으로 후두엽의 표면 중앙을 따라 이어진 calcarine sulcus에 있음. V1은 브로그만 영역의 BA17과 동일함\ref{fig:brodmann_map}\index{V1}
%  \begin{center}
%    \includegraphics[width=0.5\textwidth]{image/V1}
%    \captionof{figure}{V1}
%    \label{fig:V1}
%  \end{center}
}

\subsection{Brodmann area(BA)\index{Brodmann area}}
Korbinian Brodmann이 1909년 제작한 뇌 지도.
세포 모양, layering, 밀도 등 다양한 해부학적 특징에 기반하여 제작함.
$\Rightarrow$ 각 브로드만 영역은 다른 역할(e.g., 특정 인지 처리)을 할 수 있음.\\
(주의! 실제로 뇌 처리는 매우 복잡해서 여러 뇌 영역이 연관되고 상호작용할 수 있음(chapter 11).
$\Rightarrow$ 그럼에도 각 영역이 분명히 전문화 되어있음)

\begin{figure}[h]
  \centering
  \includegraphics[width=0.5\textwidth]{image/brodmann_map}
  \caption{Brodmann map (1909). The left hemisphere with Brodmann areas labeled
(lateral view, occipital pole to the right).}
  \label{fig:brodmann_map}
\end{figure}

\begin{table}[htb]
	\centering
	\begin{tabular}{|c|c|}
		\hline
		\thead{브로드만 영역} & \thead{관련된 것} \\
		\hline
    BA17 & V1 \\
    BA39 & angular gyrus \\
    BA40 & supramarginal gyrus \\
    lateral part of BA7 & superior parietal lobule \\
    medial part of BA7 & precuneus \\
    BA4 \& BA6 & motor processing regions \\
		\hline
	\end{tabular}
	\caption{각 브로드만 영역의 매핑}
\end{table}

superior parietal lobule과 precuneus는 기억과 연관되어 있다고 함.\footnote{비단, 두 용어 모두 책서 이후에 등장 아니함}

\begin{figure}[h]
  \centering
  \includegraphics[width=0.7\textwidth]{image/brodmann_mapping}
  \caption{Brodmann map, gyri, and sulci}
  \label{fig:brodmann_mapping}
\end{figure}

gyri/sulci와 Brodmann area는 뇌의 각 활성화와 관련되어 있음.


\section{The Hippocampus and Long-Term Memory}
patient H. M.(Henry Molaison, 1950Y 29sar)는 간질 발작으로 해마를 포함한 내측 측두엽(양쪽 반구서) 제거 수술을 함.
참고: Figure \ref{fig:HM}
지능과 성격에는 문제가 없었지만 long-term memory가 손실되는 기억상실증에 걸림.\index{patient H. M.}
\begin{figure}[h]
  \centering
  \includegraphics[width=0.7\textwidth]{image/HM}
  \caption{Depiction of medial temporal lobe resection in patient H. M.}
  \label{fig:HM}
\end{figure}

\begin{itemize}
  \item 수술 전 최근 몇년 기억 $\Rightarrow$ 거의 없음(retrograde amnesia)
  \item 수술 후 기억 $\Rightarrow$ 거의 없음(anterograde amnesia)
  \item 이전 사건 기억 $\Rightarrow$ 존재\\
    $\longrightarrow$ working memory 존재. e.g., 단어나 3자리 수를 몇분 동안 기억 가능
\end{itemize}

$\Longrightarrow$ 해마와 주변 피질은 long-term memory에 중요한 역할을 하는구나!\\

the key stages of long-term memory include \textbf{encoding}, \textbf{storage}, and \textbf{retrieval}.
The hippocampus has been associated with both long-term memory encoding and long-term memory retrieval.
Long-term memory storage depends on a process called
memory consolidation, which refers to changes in the brain regions,
including the hippocampus, underlying long-term memory (see
Chapter 3).\\
$\Longrightarrow$ long-term memory와 관련된 3가지 key stage 모두 해마와 관련되어 있음.

\section{Sensory Regions}
\label{ch1:sec:sensory_regions}
\paragraph{sensory reactivation hypothesis}
과거 기억 떠올리면 관련 지각과 관련한 뇌 영역이 활성화된다는 가설
e.g., `어제 무엇을 먹었지?' $\Rightarrow \mathfrak{Blueberry}\;\;\mathfrak{Yogurt} \Rightarrow$ 보라색(contents of memory) $\Rightarrow$ 시각? V1\index{sensory reactivation hypothesis}

visual processing, language/auditory processing, motor processing, and olfactory processing...\\
$\rightarrow$ 비단 시각 정보 많이 중요

\begin{figure}[h]
  \centering
  \includegraphics[width=0.7\textwidth]{image/sensory_regions}
  \caption{Sensory brain regions of interest.}
  \label{fig:sensory_regions}
\end{figure}

Figure \ref{fig:sensory_regions}는 시각 처리와 관련된 뇌 영역을 보여주고 있음.

전체적인 경로:
\begin{enumerate}
  \item \textbf{V1(또는 striate cortex\footnote{striate: 줄무늬, 염색되면 줄무늬 있는 것 처럼 보인다고 함.})}: 특징 처리, shape, color, location, and motion\index{striate cortex}
  \item \textbf{extrastriate cortex\footnote{striate cortex에 추가되어 있다고 함.}}: 처리하는 대상에 따라 영역 구분 가능\index{extrastriate cortex}\\
    \begin{itemize}
      \item \textbf{ventral visual regions}: 뇌의 아래쪽, 물체 정체성 처리
      \item \textbf{dorsal visual regions}: 뇌의 위쪽, 물체 위치 처리
    \end{itemize}
\end{enumerate}

2가지 경로:
\begin{itemize}
  \item \textbf{what pathway}:
    V1 $\longrightarrow$ ventral extrastriate cortex $\longrightarrow$ ventral temporal cortex
  \item \textbf{where pathway}:
    V1 $\longrightarrow$ dorsal extrastriate cortex $\longrightarrow$ parietal cortex
\end{itemize}
경로를 따라 올라갈수록 점점 높은 수준의 처리가 진행됨. 왼쪽 눈에 들어온 것은 오른쪽 반구 right visual field서, 오른쪽 눈에 들어온 것은 왼쪽 반구 left visual filed 처리되어.\index{right visual field}\index{left visual field}\\

left visual field \& right visual field $\longrightarrow$ contralateral visual processing(or left early visual areas \& right early visual areas)\\

Extrastriate cortex는 다양한 시각적 특징 처리에 특화된 영역들을 포함함:
\begin{itemize}
  \item \textbf{lateral occipital complex (LOC)}: 모양 처리\index{lateral occipital complex}\index{LOC}
  \item \textbf{V8}: 색상 처리\index{V8}
  \item \textbf{MT}: 움직임 처리\index{MT}
  \item \textbf{fusiform face area (FFA)}: 얼굴 처리\index{fusiform face area}\index{FFA}
  \item \textbf{parahippocampal place area (PPA)}: 문맥(+ places or scenes) 처리\index{parahippocampal place area}\index{PPA}
\end{itemize}
(주의! 하나의 영역이 특정 역할의 유일한 영역인 것은 아님. FFA 말고도 얼굴 처리와 관련된 영역은 많이 있다고 함.\footnote{`얼굴 처리'와 같은 역할은 그저 우리가 생각한 역할인거 아닐까요오??})\\

fMRI로 확인시 기억과 관련된 공간이 지각 관련 공간보다 작음 $\Longrightarrow$ 기억이 지각보다 상세하지 아니하기 때문이겠구나!


\section{Control Regions\index{Control Regions}}
control regions는 explicit memory 구성을 안내함.\\
기억 통제와 관련된 2가지 영역: the dorsolateral prefrontal cortex and the parietal cortex.
\begin{itemize}
  \item \textbf{dorsolateral prefrontal cortex}: 기억 결정에 관여
  \item \textbf{parietal cortex}: 기억 집중에 관여
\end{itemize}

\begin{definition}[top-down interaction]
  top-down interaction은 감각 영역을 조절하는 것. e.g., 시각적 자극에 집중하는 것.
\end{definition}

\subsection{fMRI를 통한 기억 통제 영역 찾기 실험}
\begin{center}
  \small
  \begin{spacing}{0.8}
    \begin{BVerbatim}

            +------+      +--------------------+
            | word | ---> | pictures or sounds |------.
            +------+      +--------------------+       \
                            ^                          |
                | |         \__ 1 times or 20 times __/
                v v
                 v
        seen or heard or new

    \end{BVerbatim}
  \end{spacing}
\end{center}
단어를 제시하고 1번 또는 20번 관련된 사진이나 소리를 제시함.\\
$\Rightarrow$ 1번 제시한 경우 이를 기억하기 위해서는 더 많은 통제가 필요함\\
$\Rightarrow$ fMRI로 dorsolateral prefrontal cortex와 parietal cortex가 활성화 된 것을 보임


\subsection{fMRI를 통한 item memory, source memory 영역 찾기 실험}
\begin{figure}[h]
  \centering
  \includegraphics[width=0.7\textwidth]{image/item_source}
  \caption{Item memory and source memory paradigm and fMRI results.}
  \label{fig:item_source}
\end{figure}

실험 청사진:
\begin{itemize}
  \item \textbf{제시}: 말로 설명할 수 없는 추상적인 도형이 왼쪽 또는 오른쪽으로 움직임
  \item \textbf{item memory}: `old' or `new'로 대답, 단지 친숙한가의 문제임
  \item \textbf{source memory}: `left' or `right'로 대답, 도형의 첫 위치를 떠올려 보아
\end{itemize}

$\Rightarrow$ item memory는 노란색 부분, source memory는 빨간색 부분서 활성화 되었음

\section*{Review Questions}
\tbox{
  \begin{enumerate}
    \item explicit memory와 implicit memory의 차이는?
    \item recollection와 familiarity의 차이는?
    \item 기억과 관련된 3가지 뇌 영역은?
    \item 영역 V8은 색상과 모션 중 어떤 처리 담당?
    \item dorsolateral prefrontal cortex는 감각 영역? 아니면 통제 영역?
  \end{enumerate}
}


\end{document}
