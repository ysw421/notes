\documentclass[../note.tex]{subfiles}

\begin{document}
\setcounter{chapter}{2}
\chapter{Brain Regions Associated with Long-Term Memory}

\tbox{
  정정. 미안해요ㅠㅠㅠㅠㅠ\\
  item memory \& context memory(Subsection \ref{subsec:contextmemoryitemmemory})에서...\\
  \ttbox{
    기존:
    \begin{itemize}
      \item{\textbf{context memory}}
        item은 2가지 문맥(context) 중 하나로 제시됨.\\
        e.g., 보라 \& 민트, 왼쪽 \& 오른쪽\\
        old와 new를 인식(recognition)하는 것에 기반하ㅁ\\
        e.g., 사과를 본적 있음???????
      \item{\textbf{item memory}}
        item에서 문맥을 회상(recall)함\\
        e.g., 그 사과가 무슨 색이었었었지?
    \end{itemize}
  }\\
  \ttbox{
    수정:
    \begin{itemize}
      \item{\textbf{study phase}}:\\
        항목(item)은 2가지 상황(context) 중 하나로 제시됨.\\
        e.g., 보라 \& 민트, 왼쪽 \& 오른쪽
      \item{\textbf{item memory}}:\\
        old와 new를 인식(recognition)하는 것에 기반함\\
        e.g., 사과를 본적 있음???????
      \item{\textbf{context memory}}:\\
        item에서 문맥을 회상(recall)함\\
        e.g., 그 사과가 무슨 색이었었지?
    \end{itemize}
  }\\
  {\small
  During the study phase of such experiments, items are presented in one of two contexts, such as on the left or right side of the screen or in red or green.
  }
}\\

\begin{note}{되새기기}\\
  long-term memory서...\\
  \begin{center}
    \begin{tabular}{|l|l|l|}
      \hline
      \thead{관점} & \thead{용어 쌍} & \thead{설명} \\
      \hline
      과제 수행 & Context Memory vs Item Memory & 실험에서 측정하는 행동 수행 \\
      주관적 경험 & ``Remembering'' vs ``Knowing'' & 참가자가 느끼는 주관적 체험 \\
      일반적 개념 & Recollection vs Familiarity & 포괄적인 이론적 개념
      \hline
    \end{tabular}
    \captionof{table}{claude 제공}
  \end{center}\\
\end{note}

\section{Episodic Memory\index{episodic memory}}
can refer to many other related
forms of memory including\\
context memory, source memory,\\
“remembering,”(the subjective experience during
detailed retrieval, e.g., '차 어디 주차?' 상상하는 시각 경험)\\
recollection(accurate retrieval of contextual informa-
tion, e.g., 어느쪽에 주차되어 있었는지),\\
and autobiographical memory(a speci c type of
episodic memory for detailed personal events)\\

episodic memories는 control regions
and sensory regions 모두와 관련되어 있음. 비단, 본 chapter서는 sensory region은 다루지 아니함.\\
episodic memory와 관련된 control regions:\\
medial temporal lobe(which include \textbf{the hippocampus and the parahippocampal cortex}),\\
dorsolateral prefrontal cortex(이미지 가장 왼쪽, 활성화 아니 된 곳),\\
parietal cortex(inferior
parietal cortex (\textbf{angular gyrus}) and the \textbf{medial parietal cortex} (extending
into \textbf{retrosplenial cortex and posterior cingulate cortex}: 함께 활성화))\\
\textbf{Bold}된 글씨는 Figure \ref{fig:episodic_regions}에서 라벨링된 구역임.
\begin{figure}[htbp]
  \centering
  \includegraphics[width=0.8\textwidth]{image/episodic_regions}
  \caption{Regions of the brain associated with episodic memory. fMRI activity (in red/yellow)
  in the left hemisphere}
  \label{fig:episodic_regions}
\end{figure}

\subsection{the hippocampus and the parahippocampal cortex\index{hippocampus}\index{parahippocampal cortex}}
비교적 잘 알려진 지역.
\subsubsection{parahippocampal cortex}
processes the context of previously presented information: \\
e.g., the spatial location (e.g., the left or right side of the screen) \\
or the color (e.g., red or green) of an object that was presented earlier.\\
시간적 정보 처리한다는 fMRI 증거 있음.

\begin{note}{experiment (Ekstrom, Copara, Isham, Wang & Yonelinas,
  2011)\footnote{\url{http://humanspatialcognitionlab.org/wp-content/uploads/2014/10/EkstromCopara\_etal2011\_Neuroimage-1.pdf}}}\\
  실험 청사진:
  \begin{itemize}
    \item \textbf{study phase}:\\
      참가자들을 가상환경서 8개의 다른 매장 위치와 방문 순서를 학습함.
    \item \textbf{test phase}: \\
      하나의 목표 매장을 선택 받은 후...
      \begin{itemize}
        \item 가장 가까운 다른 매장 2개(spatial)
        \item 가장 방문 순서가 인접한 2개(temporal)
      \end{itemize}
      선택.
  \end{itemize}
  $\Rightarrow$ 정확한 공각/시간 기억은 parahippocampal cortex를 활성화함.
\end{note}

\subsubsection{perirhinal cortex\index{perirhinal cortex}}
parahippocampal cortex의 앞쪽, medial temporal lobe에 속함.\\
item information(e.g., 이전에 본 적이 있는지?) 처리 $\Rightarrow$ episodic memory는 item들로 구성되어 있음.\\
그저 familiarity?

\subsubsection{hippocampus}
여러개 item + context 묶음 $\Rightarrow$ 상세한 episodic memory
\begin{figure}[htbp]
  \centering
  \includegraphics[width=0.3\textwidth]{image/hippocampus}
  \caption{Model of medial temporal lobe sub-region function. The perirhinal cortex (PRC)
    processes item information, the parahippocampal cortex (PHC) processes context
    information, and the hippocampus (HC) binds item information and context information.}
  \label{fig:hippocampus}
\end{figure}
e.g., 캘리포니아 Newport 해변으로 여행간 기억을 떠올림 $\rightarrow$ perirhinal cortex: 친구, parahippocampal cortex: 서있던 지역

\subsection{dorsolateral prefrontal cortex \& parietal cortex}
역할 잘 알려진 바 없음.(! 연구의 주요 주제)\\
dorsolateral prefrontal cortex 추측: 1) 기억의 내용을 평가 2) 다른 영역에 저장된 정보 선택\\
parietal cortex 추측: 1) 기억 내용 저장 2) attention\\
inferior parietal cortex: ``remembering''과 관련, 비단 source memory와 관련 없음.참조: Figure \ref{fig:claude_inferior}\\
\begin{figure}[htbp]
  \centering
  \includegraphics[width=0.7\textwidth]{image/claude_inferior}
  \caption{@claude}
  \label{fig:claude_inferior}
\end{figure}

\begin{note}{long-term memory의 regions}
  episodic memory 관련 영역: the dorsolateral prefrontal cortex, the parietal cortex, and the medial temporal lobe\\
  $\Longrightarrow$ item memory와도 관련되어 있음 $\Longrightarrow$ 궁극적으로 long-term memory와 관련된 지역임.
\end{note}

\section{Semantic Memory\index{semantic memory}}
Semantic memory has been associated with the\\
left dorsolateral prefrontal cortex (episodic memory와 다른 영역),\index{left dorsolateral prefrontal cortex}\\
the anterior temporal lobes,\index{anterior temporal lobe}\\
and sensory cortical regions\index{sensory cortical regions}\\

\paragraph{left dorsolateral prefrontal cortex}
언어 처리 또는 다른 피질 영역에 저장된 sementic memory 선택 처리

\paragraph{sensory cortical regions}\\
\begin{itemize}
  \item \textbf{동물 이름 말하기}
    $\rightarrow$ lateral inferior occipital-temporal cortex 활성화(living things 인식하는)
  \item \textbf{도구 이름 말하기}
    $\rightarrow$ medial inferior occipital-temporal cortex 활성화(nonliving things 인식하는)
\end{itemize}

\paragraph{anterior temporal cortex}\\
알츠하이머 환자를 대상으로 left anterior temporal lobe가 sementic memory와 관련 있음을 알게 되었음.
\begin{figure}[htbp]
  \centering
  \includegraphics[width=0.3\textwidth]{image/anterior_temporal}
  \caption{Regions of the brain associated with semantic memory. Cortical thinning in
Alzheimer’s patients (in red/yellow) associated with disruption in semantic memory.}
  \label{fig:anterior_temporal}
\end{figure}

anterior temporal cortex 역할: ? ! 쵠근 연구 주제\\
추측: 1) sementic information 저장 2) 다른 피질 영역에 저장된 정보를 연결(마치 해마가 episodic memory 연결하는 것 처럼)\\

anterior temporal cortex가 여러 감각(시각, 청각, 사회적)의 sementic memory를 매개한다는 증거가 있음.

\begin{note}{experiment(Simmons, Reddish, Bellgowan \& Martin, 2010)}\\
  실험 청사진:\\
  피험자는 사람, 건물, 망치에 대하여 학습함: `the brooks hammer is eight years old'. `patrick was born in little rock'\\
  반복학습하며 fMRI 관측 $\Longrightarrow$ 사람 학습시 유독 left and right anterior temporal lobe 활성화\\
  $\Rightarrow$ anterior termporal cortex는 social information과 관련있지 아니할까?
\end{note}

\section{Memory Consolidation\index{memory consolidation}}
표준 가설: long-term memory: hippocampal–cortical interactions $\Longrightarrow$ cortical–cortical interactions (1-10년 소유)\\
증거: 해마 손상된 기억 상실증 환자, 1년 전을 기억하네, 해마는 그 보다 오래된 기억에 관여하지 아니하나?\\
비단, 해마 손상되니 30년전 기억 손상된 사례가 존재함..((Nadel \& Moscovitch, 1997))\\
대안 가설: 해마는 평생 장기 기억 검색에 관여?\\

\begin{figure}[htbp]
  \centering
  \includegraphics[width=0.3\textwidth]{image/alternative_model}
  \caption{@claude ❤️ }
  \label{fig:alternative_model}
\end{figure}

\begin{itemize}
  \item 부시 대통, 대량 학살 무기 있다는 추측으로 어느 나라에 전쟁 선포했는가?\\
    $\rightarrow$ semantic memory $\rightarrow$ 해마 상관 없음, 기억 가능
  \item 15년전 유니버셜 스튜디오 방문한 적이 있는가?\\
    $\rightarrow$ autobiographical memory $\rightarrow$ 기억 불가
\end{itemzie}

오래된 기억일 수록 회상할 때 해마의 영향이 작다는 연구(당시 뉴스를 물어보며 fMRI 측정)\\
비단, 문제점 두가지:
\begin{enumerate}
  \item Semantic memory으로 실험\\
    $\rightarrow$ 해마는 episodic memory에 더 중요, 잘못된 실험 아닐까?
  \item 해마 활동이 0으로 떨어지지 아니함
\end{enumerate}

해마가 autobiographical memory에 중요함을 확인해 보자.
TGA(transient global amnesia) 환자는 해마만 일시적으로 손상된 환자(focal lesions restricted to the hippocampus). Figure \ref{fig:TGA} 참조.\footnote{https://pmc.ncbi.nlm.nih.gov/articles/PMC3198338/}\\

\begin{figure}[htbp]
  \centering
  \includegraphics[width=0.3\textwidth]{image/TGA}
  \caption{Autobiographical memory disruption for recent and remote events in patients
with hippocampal lesions. Autobiographical memory score plotted as a function of the time
period before the onset of transient global amnesia (TGA). TGA patient performance (in gray)
and control performance, including performance when the TGA patients no longer had
a lesion (follow-up, in black) and performance of participants with no lesions (in white; key to
the right).}
  \label{fig:TGA}
\end{figure}

? 환자 14명 p-value 0.001?\\

consolidation 연구가 어려운 이유:
\begin{itemize}
  \item 순수히 해마만 손상된 사람 찾기 어려움
  \item 실험하는 과정서 해마가 사용되는 모순적 상황 자주 등장
\end{itemize}

\section{Consolidation and Sleep}
알려지길 consolidation은 다음 수면서 시작되어.\\
기본 규칙: 옛 기억 방해 최소화 하며 새로운 기억을 통합함.

수면은 REM(rapid eye movement) 수면과 non-REM(NREM) 수면으로 나뉨.\index{REM sleep}\index{non-REM sleep}\\
\begin{figure}[htbp]
  \centering
  \includegraphics[width=0.8\textwidth]{image/sleep_p}
  \caption{REM and non-REM (NREM) sleep stages as a function of time for
a typical night of sleep (thick gray bars show REM sleep).}
  \label{fig:sleep_p}
\end{figure}

\begin{itemize}
  \item \textbf{REM 수면}:
    수면 중 두번째 반(second half)서 주로 나타남, implicit memory consolidation에 중요한 것으로 보임
  \item \textbf{non-REM 수면}:
    4개로 나뉘며, 3, 4단계(slow wave sleep)는 long-term memory consolidation에 중요함.\index{slow wave sleep}
\end{itemize}

+ Neuroinsight workshop서 사용한 최신 논문: NREM은 쥐 기준 microstructer로 나뉘어져 있으며,
수축된 동공 상태서는 최근 기억을 재생 후 consolidation하며, 확장된 동공 상태서는 오래된 기억을 재생 후 consolidataion\\

\subsection{slow wave sleep\index{slow wave sleep}}
느린 파(less than 1 Hertz)\\
down-states(뇌의 전반적 활동 감소)와 up-states가 번갈아 나타남.

\subsection{파들}
Figure \ref{fig:waves} 참고

\begin{figure}[htbp]
  \centering
  \includegraphics[width=0.7\textwidth]{image/waves}
  \caption{Schematic of the brain that
includes the cortex (off white), the thalamus (dark gray structure near the center), and the
hippocampus (light gray structure; occipital pole to the left) in addition to cortical slow waves,
thalamic-cortical sleep spindles, and hippocampal sharp-wave ripples (labels to the right).}
  \label{fig:waves}
\end{figure}\\

\begin{itemize}
  \item\textbf{thalamic-cortical sleep spindle}:\index{thalamic-cortical sleep spindles}\\
    11–16 Hertz로 진동

  \itme\textbf{Hippocampal sharp-wave ripples(Hippocampal SWR)}:\index{Hippocampal sharp-wave ripples}\\
    대략 200 Hertz로 진동\\
    기억을 재생해서 공고화 $\Rightarrow$ 중요! $\Longrightarrow$ 2가지 실험 제시하지만 생략.
\end{itemize}

\section{Memory Encoding\index{encoding}}
memory encoding은 기억 획득 과정 $:<$

\subsection{subsequent memory analysis\index{subsequent memory analysis}}
실험실서 memory encoding과 관련된 영역 찾기 위해 subsequent memory analysis를 사용.

\begin{enumerate}
  \item \textbf{study phase}: item list 보여주기\\
    e.g., `wolf', `ocean'이 포함된 list 보여주기
  \item \textbf{test phase}: old, new item 보여주고 피험자 ``old'' 또는 ``new'' 판단하기\\
    e.g., 피험자가 `wolf'는 ``old''라고 판단하고, `ocean'은 ``new''라고 판단함.
    $\longrightarrow$ study phase서 wolf와 ocean 각각 뇌가 어떻게 반응하였더라??
\end{enumerate}

memory encoding와 retrieval은 유사함 $\rightarrow$ 영역도 유사함.\\
sensory regions and control regions, including the dorsolateral prefrontal
cortex, the parietal cortex, and the medial temporal lobe: 참고: Figure \ref{fig:encoding_region}

\begin{figure}[htbp]
  \centering
  \includegraphics[width=0.6\textwidth]{image/encoding_region}
  \caption{Regions of the brain associated with subsequent memory effects. fMRI
    activations associated with subsequent memory (in red/yellow; top, lateral views, occipital
    poles toward the center; bottom, coronal views, the left image is the most anterior and the
    right image is the most posterior). Medial temporal lobe activity, centered on the
    hippocampus, is shown near the bottom of each coronal image in both hemispheres.}
  \label{fig:encoding_region}
\end{figure}\\

당연히! 세부적인 것 달라:\\
e.g., 1) perirhinal cortex 활성화 규모 다른데, 이후 chapter들서 많이 다룸.\\
2) 해마의 하위 영역 활동 패턴이 다름.

\section{Sex Differences}
남자: spatial memory 우수(이전에 학습한 환경 탐색, 기억에 잠시 유지하는 working memory가 필요할 때)\\
여자: verbal memory 우수(단어 인식, 회상, 연상, autobiographical memory)\\
$\rightarrow$ 가상 미로서 위치 외우는 실험서도 서로 다른 전략 사용\\

대부분 long-term memory가 verbal memory를 사용하여 behavioral performance가 일반적으로 여자가 더 우수함.\\
the hippocampus and the dorsolateral prefrontal cortex에 estrogen receptors 많고, 뇌 비율 중 많이 차지함.

\section{Superior Memory}
\begin{figure}[htbp]
  \centering
  \includegraphics[width=\textwidth]{image/superior_memory}
  \caption{세상은 아름다워 @LM}
  \label{fig:superior_memory}
\end{figure}

\begin{figure}[htbp]
  \centering
  \includegraphics[width=0.5\textwidth]{image/von}
  \caption{폰 노이만(John von Neumann)의 기억력 관련 일화들은 정말 전설적입니다! 😲
 @claude}
  \label{fig:von}
\end{figure}

알려진 것이 거의 없음. $\longrightarrow$ 해당하는 사람을 피험자로 섭외하기 어려움.\\

런던의 택시 기사들 상대로 연구해 보아. posterior hippocampus의 gray matter 자란 반면, anterior hippocampus의 gray matter은 감소함.

\epigraph{블랙캡 운전사는 유일한 자격시험으로 지식 테스트를 넘어야 한다. 이를 통과하려면 런던 지리에 관한 모든 지식이 필요하며 합격에 필요한 학습 시간만 해도 평균 잡아도 무려 4년에 달한다.}{--- @www.techholic.co.kr}

\begin{figure}[htbp]
  \centering
  \includegraphics[width=0.5\textwidth]{image/taxi}
  \caption{Change in the size of the posterior hippocampus as a function of time as a
London taxi driver. The size of the posterior hippocampus (shown on the y-axis) was
measured using voxel-based morphology (VBM).}
  \label{fig:taxi}
\end{figure}

?ㅋ\\

superior memory 군들:
\begin{itemize}
  \item 런던 택시 기사
  \item 세계 기억력 챔피언들
  \item $\pi$ 암송 능력자
  \item highly superior autobiographical memory, 참고: Figure \ref{fig:superior_memory}\index{highly superior autobiographical memory}
\end{itemize}
\\

\vspace{1em}
특징들:
\begin{itemize}
  \item 평균 지능과 비슷
  \item Zero-sum Model: 한 영역 향상 $\rightarrow$ 다른 영역 저하
\end{itemize}

\section{Review Questions}
\begin{itemize}
  \item What are the brain regions most commonly associated with episodic
memory?
  \itme Is there more evidence supporting the standard model of consolidation or
the alternative model of consolidation?
  \item Which type of sleep is particularly important for long-term memory
consolidation?
  \item Are long-term memory encoding and retrieval relatively similar
cognitive processes or relatively different cognitive processes?
  \item Which strategy do females typically employ to a greater degree than
males during long-term memory tasks?
  \item Do those with superior memory have advanced abilities on most
cognitive tests?
\end{itemize}



\end{document}
