\documentclass[../note.tex]{subfiles}

\begin{document}
\chapter{Brain Timing Associated with Long-Term Memory}

temporal dynamics of activity $\rightarrow$ understand the brain mechanisms underlying memory

\section{Timing of Activity}
ERP\\
naming: parietal (P) or frontal (F), odd: left hemisphere, even: right hemisphere\\
high voltage $\rightarrow$ cortical activity\\

ERP studies of memory have focused on two brain activity: to re ect familiarity and recollection\\

ERP component 1) mid-frontal old–new effect\\
  occurs within 300 to 500 milliseconds after trial onset\\
  maximum amplitude over frontal electrodes\\
  than correct rejection of
new items 보다 familiarity-based retrieval서 큰 진폭\\

ERP component 2) left-parietal old–new effect\\
occurs within 500 to 800 milliseconds after trial
onset \\
a maximum amplitude over left parietal electrodes\\
correct rejection of new
items 보다 recollection-based retrieval서 큰 진폭\\


FN400: frontal negative 400ms, 자극 후 전두엽서 400ms 후 나오는 음 값\\

``remember'' 등급: R1(덜 상세), R2(상세)\\
activity at electrode P5 within 500 to 800 milliseconds: R2 > R1 > new\\
$\rightarrow$ magnitude of the left-parietal old–new effect reflects the amount of information retrieved during recollection.\\

오랜 논쟁: familiarity and recollection가 과정이 다른가?\\
mid-frontal old–new effect and the left-parietal old–new effect:\\
topographically separable, temporally separable, and functionally separable\\
$\Longrightarrow$ 다른 듯!\\

third ERP component ) right-frontal old–new effect\\
occurs within 1000 to 1600 milliseconds\\
maximum amplitude over right frontal electrodes\\
correct rejection of new items 보다 recollection-based or familiarity-based retrieval of old items서 큰 진동\\
일반적으로 주목하지 아니해.\\
혹시? 추측! : post-retrieval
monitoring (i.e., evaluating what was just remembered) or memory
elaboration (i.e., filling in details of the previous experience)\\

\section{The FN400 Debate}
mid-frontal old–new effect의 FN400\\
가설: repetition priming 반영: change in the magnitude of brain activity that occurs when an item is repeated\\
Amnesic patients(with medial temporal lobe damage: impaired conscious long-term memory) $\rightarrow$ normal repetition priming effects
$\rightarrow$ is a nonconscious process?????\\

Paller왈, FN400는 priming referred의 한 유형: conceptual repetition
priming: 틀린 주장\\
Figure 4.2: conceptual priming effects: 400ms부터 high meaning old가 덜 떨어짐, 이는 FN400가 다름.\\
위치도 달라\\
mid-frontal old–new effect $\rightarrow$ familiarity

\subction{Phase and Frequency of Activity}
Synchronized brain activity: 실험, 이전 새로운 도형, 오른족 시야 기억 - 왼쪽 시야 기억, 그리고 old-left-hits were subtracted from old-right-hits\\
같은 반구서 동일한 시간에 활성화? 맞음!\\
Such synchronous activity is referred to
as in phase or phase-locked\\
심지어, 전두, 측두, 후두도 동기화\\

Figure \ref{fig:waves2}\\
\begin{figure}[htbp]
  \centering
  \includegraphics[width=0.6\textwidth]{image/waves2}
  \caption{82쪽 중간 부분: @claude}
  \label{fig:waves2}
\end{figure}

\subsection{cross-frequency coupling}
brain regions with different frequencies of modulation can be in phase with each other $\rightarrow$ two brain regions interact\\

파 활동들 important role during long-term memory encoding and retrieval\\
Theta activity $\rightarrow$ frontal regions\\
Gamma activity $\rightarrow$ parietal-occipital regions\\



\end{document}
