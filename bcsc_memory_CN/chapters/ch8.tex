\documentclass[../note.tex]{subfiles}

\providecommand{\mainfile}{standalone}

\ifx\mainfile\empty\else
    \usepackage{import}
    \title{Chapter 8 - Memory and Other Cognitive Processes}
    \author{BCSC}
    \date{\today}
\fi

\begin{document}
\ifx\mainfile\empty\else
    \maketitle
    \newpage
\fi

\setcounter{chapter}{7}
\chapter{Memory and Other Cognitive Processes}

이 장에서는 memory의 인지 과정과 뇌 영역과 attention\footnote{Attention은 모든 explicit memory의 
내용에 집중된다.}, imagery, language, and emotion의 인지 과정과 뇌 영역을 비교한다.

\begin{itemize}
  \item Section \ref{sec:attention}: attention
  \item Section \ref{sec:imagery}: imagery
  \item Section \ref{sec:language}: language
  \item Section \ref{sec:emotion}: emotion
\end{itemize}


\section{Attention and Memory\index{attention}}
\label{sec:attention}
Behavioral research: attention은 item의 처리 과정을 향상 시킴 $\longrightarrow$ 정확하고 빠른 반응.\\
e.g., 화살표를 통해 attention할 부분을 표시(L | R) $\rightarrow$ 표시된 부분을 더 빠르게 탐지\\

\paragraph{관련 영역}
sensory regions and control regions

\begin{definition}{gain model of attention}\\
  attention이 감각 처리 영역의 활동의 강도를 증진시키는 model
\end{definition}

gain model of attention에서는 attention이 item의 처리 과정을 향상 시키는 것의 원인이 될 수 있다.\\

\tbox{sensory effects of attention의 실험 방법}{
  양쪽 visual field를 동시에 자극함으로써 연구한다.
  피험자는 항상 중앙을 바라보며 왼쪽 또는 오른쪽에 제시된 것을 attention하라는 지시를 받는다.
  \begin{center}
    \includegraphics[width=0.6\textwidth]{image/ch8_blueprint1}
    \captionof{figure}{Spatial attention paradigm and fMRI results. (A) Attention stimulus display with
two overlapping arrowheads at the central fixation point and a flashing checkerboard
stimulus within each visual field. When one arrowhead briefly turns red, participants shift
attention to the corresponding visual field/stimulus (illustrated by the dotted circle).
Participants press a button when they detect a small red square within the attended location/
stimulus and ignore the unattended location/stimulus. (B) Contralateral attention activity in
early visual regions (axial view, occipital pole at the bottom). The contrast of attention to the
right visual field and attention to the left visual field (Right > Left) produced activity in the left
extrastriate cortex (in purple/cyan), while the contrast of attention to the left visual field and
attention to the right visual field (Left > Right) produced activity in the right extrastriate
cortex (in red/yellow). (C) Attention control activity in the dorsolateral prefrontal cortex (the
rightmost activation) and the parietal cortex (the leftmost activation) of the right hemisphere
(in purple/yellow; lateral-posterior view, occipital pole to the left).}
    \label{fig:ch8_blueprint1}
  \end{center}
  \paragraph{청사진(Figure \ref{fig:ch8_blueprint1}-A의 각 panel 순으로):}
  \begin{enumerate}
    \item
      중앙에 겹쳐진 화살표(L, R)가 존재한다. 이 중 하나가 붉은색\footnote{claude 왈, 붉은색: V8 영역(색상 처리 영역)에서 strong response}으로 변하면, 피험자는 해당 부분에 attention한다.
    \item
      양쪽에 체크보드(!! Chapter 6)가 존재한다. attention하고 있는 부분서 가끔 체크보드가 붉은색으로 변하면 피험자는 버튼을 누른다.
    \item
      다른 화살표가 붉은색으로 변하였다. 이는 피험자가 새로운 부분에 attention하라는 지시이다.
    \item
      피험자는 지시에 따라 새로운 부분에 attention한다.
  \end{enumerate}

  \paragraph{fMRI 결과(Figure \ref{fig:ch8_blueprint1}-B):}
  \begin{itemize}
    \item
      왼쪽 visual field에 attention할 때, 오른쪽 extrastriate cortex가 활성화 된다.
    \item
      오른쪽 visual field에 attention할 때, 왼쪽 extrastriate cortex가 활성화 된다.
  \end{itemize}
  $\Longrightarrow$ contralateral attention effects(early visual regions, including V1, V2, and V3) 
  $\approx$ contralateral attention effects in visual processing regions(Chapter 1)\\

  + attention은 특징 영역을 강화함: e,g., color $\longrightarrow$ the color processing region within the ventral visual processing stream

  \paragraph{Thakral & Slotnick, 2009의 실험}\\

  \url{https://www.sciencedirect.com/science/article/pii/S0006899309019337?dgcid=api\_sd\_search-api-endpoint}
  \begin{center}
    \includegraphics[width=0.6\textwidth]{image/ch8_bp2}
    \captionof{figure}{The flicker protocol was a circular
checkerboard that reversed in contrast during attention and perception periods, with no flicker during stationary periods.
During attention periods, participants were instructed to press a button each time a red square briefly appeared.}
    \label{fig:ch8_bp2}
  \end{center}
  \paragraph{Attention과 perception 비교(Figure \ref{fig:ch8_blueprint1}-C):}
  dorsolateral prefrontal cortex and parietal cortex control regions(Chapter 6: working memory와 동일)\\
  working memory contralateral visual sensory effects $\approx$ spatial attention(L \| R)과 관련된 contralateral visual sensory effects
}

\subsection{working memory와 attention}
DL-PFC와 parietal cortex는 working memory와 attention 과정 모두에서 활성화 된다.
이는 두 인지 과정이 연결되어 있다는 증거로 사용되곤 하다.

\tbox{Ikkai & Curtis, 2011}{
  \url{https://pmc.ncbi.nlm.nih.gov/articles/PMC3081523/pdf/nihms260559.pdf}
  \begin{center}
    \includegraphics[width=0.4\textwidth]{image/ch8_bp3}
    \captionof{figure}{Task and results summary from working memory (WM), attention studies.}
    \label{fig:ch8_bp3}
  \end{center}
  delay/sustained attention period 동안, PFC와 PPC에서 비슷한 pattern이 보인다!\\
  비단, 차이점도 존재합니다. working memory, 복잡한 자극(얼굴들과 집들), 반면 attention, 아니 복잡한 자극(체크보드 패턴)\\
}

\subsection{long-term memory와 attention}
long-term memory 이야기를 해요.

\paragraph{청사진}
\begin{itemize}
  \item study phase:\\
    to-be-remembered stimuli들 있는 리스트가 있다. 이를 보여준다.
  \item test phase:\\
    학습이 끝났는가? old와 new 항목이 보여진다. 피험자는 ``old''-``new'' 판단을 한다. 경우에 따라 context memory도 판단할 수 있다.
\end{itemize}

2개의 long-term memory 종류(Episodic memory and item memory)와 관련된 sensory processing regions은 다음과 같다:
the dorsolateral prefrontal cortex, the parietal cortex, and the medial temporal lobe\\
MTL을 제외한 타 영역, attention과 동일하다!\\
$\longrightarrow$ 여러 과학자 왈, 이는 attention이 retieval할 때 attention이 item에 대하여 내부 표현(internal representation)을 조정할 수 있기 때문이다.\\
예컨대, `나는 차 키를 어디에 두었는가?' $\longrightarrow$ 키에 주의 집중한다.\\

동일한 same sensory effects에 대하여 long-term memory와 attention이 관련되어 있다는 증거를 살펴본다.
\tbox{Slotnick, 2009\footnote{\url{https://www.sciencedirect.com/science/article/pii/S0006899309004582?via\%3Dihub}}}{
  추상적인 모양을 짜극으로 하는\footnote{앞선 연구에서 verbal memory를 통해 모양을 기억하는 것을 막고자 추상적인 모양을 사용한 적이 있다!} fMRI-ERP 연구이다.
  \begin{center}
    \includegraphics[width=0.4\textwidth]{image/ch8_bp4}
    \captionof{figure}{(A) During the study phase, shapes were presented to
the left or right of fixation. (B) During the test phase, old
shapes from the study phase (previously on the left, old-left,
or previously on the right, old-right) or new shapes were
presented at fixation and participants classified each shape
as old and on the “left”, old and on the “right”, or “new”
(correct responses are shown to the right).}
    \label{fig:ch8_bp4}
  \end{center}
  \paragraph{청사진}
  \begin{itemize}
    \item study phase:\\
      central fixation point로 부터 왼쪽 또는 오른쪽에 추상적인 모양이 존재한다.
    \item test phasE:\\
      fixation point에 old 또는 new item이 나타난다. 피험자는 구분한다:
      ``old이고 이전에 왼쪽에 있었습니다'', ``old이고 이전에 오른쪽에 있었습니다'', ``$\mathfrak{new}$''.\\
      참고로 피험자에게 ``old'', ``new''처럼 언어적 전략을 활용한 기억보다 시각적 전략을 활용한 기억을 권장한다.
  \end{itemize}
  \begin{center}
    \includegraphics[width=0.8\textwidth]{image/ch8_result1}
    \captionof{figure}{(A) During the study phase, shapes were presented to
the left or right of fixation. (B) During the test phase, old
shapes from the study phase (previously on the left, old-left,
or previously on the right, old-right) or new shapes were
presented at fixation and participants classified each shape
as old and on the “left”, old and on the “right”, or “new”
(correct responses are shown to the right).}
    \label{fig:ch8_result1}
  \end{center}

  \paragraph{결과}\\
  \begin{itemize}
    \item (old-left-hits) - (old-right-hits) = right extrastriate cortex
    \item (old-right-hits) - (old-left-hits) = left extrastriate cortex
  \end{itemize}

  contralateral P1\footnote{P1: positive $\sim$100ms, FN400: frontal negative $\sim$400ms} effect가 보인다.
  즉, 자극을 주고 100에서 200ms 후에 반대쪽 반구서 반응이 발생하였다.\\

  $\longrightarrow$ 이러한 이유로 long-term memory와 attention은 동일한 메카니즘을 가진다.\\

  ??\\

  추론컨대, 책은 해당 실험서 long-term memory와 관련된 부분만을 다루고, 이전 attention 실험들과 비교하여 패러다임의 동일성을
  말하는 듯 한다.

  \begin{center}
    \includegraphics[width=0.8\textwidth]{image/ch8_result2}
    \captionof{figure}{B) Toward the center, posterior view of differential ERP voltage scalp topographies (with ROIs
delimited by black ovals) are shown illustrating encoding-related retinotopic activation at the mean timepoint within select
lateralized epochs from above (color scale at center). Corresponding occipital dipole sources underlying these voltage
topographies are also shown (dipoles in the right and left hemispheres are colored red and blue, respectively). (C) Toward the
center, spatial encoding-related fMRI activity in retinotopic visual regions projected onto a cortical surface representation
(posterior view, with the right hemisphere to the right; gyri and sulci colored light and dark gray, respectively). Retinotopic
encoding effects were isolated by contrasting encoding-left > encoding-right (significant activity shown in red) and
encoding-right > encoding-left (significant activity shown in blue). Corresponding event-related timecourses extracted from
two regions of activity are also shown.}
    \label{fig:ch8_result2}
  \end{center}

  Figure \ref{fig:ch8_result2}는 논문에 포함된 다른 Figure인데, 양쪽 영역을 contrast하는 것을 보아 attention과 관련되어 있다.
  아닌가, encoding이라고 적혀있다.\\
  
  + contralateral early visual region activity during long-term memory encoding에서도 동일한 패러다임이 보인다고 한다.
}

long-term memory가 작동하는 동안 attention이 작동하는 것처럼 보이지만, long-term memory는 MTL과 관련 있다.
따라서 둘은 독립이고, attention은 내부 기억 표현을 강화한다.
이러한 이유를 말미암아 둘의 관계를 밝히기 위해 개별 참가자의 활동을 비교하는 것이 필요함다.


\section{Imagery and Memory\index{imagery}}
\label{sec:imagery}
Stephen Kosslyn, $\mathfrak{a\;\textbf{\textit{brilliant}}\;cognitive\;neuroscientist}$는 imagery애 많은 관심이 있다.\\
`German Shephard(?)의 귀는 어떤 모양인가?'라는 질문에 사람들이 어떻게 답변할 수 있는지 궁금하다.\\
$\longrightarrow$ 대부분의 사람들: German Shephard의 visual mental image 상상하기
$\rightarrow$ 귀를 ``looking'' 하기\\
$\Longrightarrow$ V1 and extrastriate cortex regions 뿐 아니라 DL-PFC와 parietal cortex control regions도 관련 있다!

\tbox{fMRI 실험}{
  visual perception, visual imagery, and visual attention과 관련된 sensory regions and control regions 비교하기\footnote{https://academic.oup.com/cercor/article/15/10/1570/396851?login=true}\\

  \begin{center}
    \includegraphics[width=0.6\textwidth]{image/ch8_bp5}
    \captionof{figure}{Visual perception, imagery, and attention paradigms and fMRI results. (A) Left,
perception stimulus display, with flashing checkerboards rotating around the central fixation
point. Right, imagery and attention stimulus display with only the outer arcs of the flashing
checkerboards rotating around the central fixation point. During the perception and imagery
condition, participants determined whether a briefly flashed small red square was“inside”
or
“outside” of the stimuli. During the attention condition, participants determined whether the
small red square was in the“left” visual field or the“right” visual field. (B) Perception,
imagery, and attention retinotopic maps for a representative participant (posterior view;
colors correspond to different spatial locations in the visual field as shown by the semi-circle
key between the perception and imagery retinotopic maps). Early visual regions are labeled
(in black) and cyan ovals show the regions where imagery produced greater retinotopic
activity than attention. The repeating patters (?????) of colors (e.g., yellow to red to yellow to red in
the upper left hemisphere) correspond to repeated visual field representations in early visual
areas (e.g., the lower right quadrant in the visual field has a unique representation in dorsal
V1, V2, and V3 of the left hemisphere). (C) Activity associated with both perception and
imagery (in green) and activity associated with both imagery and attention (in orange; key to
the right).}
    \label{fig:ch8_bp5}
  \end{center}

  \paragraph{청사진: Figure \ref{fig:ch8_bp5}-A}
  \begin{itemize}
    \item visual perception:\\
      좌측 그림, 회전하는 체크박스가 있고 그 안에 빨간점, ``inside'' 또는 ``outside''를 판단한다.
    \item visual imagery:\\
      우측 그림(짜짜 크게 확대하여 보아), 끝 부분에 윤각 호만 존재하여, ``완전한 체크보드를 상상해 보아!''
      $\longrightarrow$ 빨간점은 ``inside'' 또는 ``outside''를 판단한다.
    \item visual attention:\\
      우짜 그림, ``절대 완전한 체크보드를 상상 안해 보아''
      $\longrightarrow$ 빨간점은 왼쪽 또는 오른쪽을 판단한다.
  \end{itemize}

  \paragraph{결과: Figure \ref{fig:ch8_bp5}-B}\\
  \begin{definition}{retinotopic maps}
    visual field에 인접한 위치에 early visual regions가 cortex에 인접한 위치에 매핑되는 활성화이다.
    이는 망막이 visual field에 매핑되는 방법이다.
  \end{definition}
  attention과 비교했을 때, imagery는 perception과 유사하다.\\
  cyan 타원은 attention서 발현되지 아니한 곳이와요, 참가자들서 전체적으로 일어나는 현상이다.\\
  $\rightarrow$ imagery는 perception과 유사하지만 더 약한 듯하다.

  \paragraph{결과: Figure \ref{fig:ch8_bp5}-C}\\
  \begin{itemize}
    \item green:\\
      perception and imagery: included visual processing regions
    \item orange:\\
      imagery and attention: included the dorsolateral prefrontal cortex and the parietal cortex
  \end{itemize}

  $\Longrightarrow$
  \begin{itemize}
    \item visual perception and visual imagery: 중복된 sensory regions
    \item visual imagery and visual attention: 같은 control regions
  \end{itemize}
}

Visual imagery와 visual working memory는 떼려야 뗄 수 없는 친구이다.\footnote{Chapter 6서 working memory는 imagery의 다른 이름일지도 모른다는 논의를 하였다!}\\
delay period를 측정하고, V1 포함한 visual sensory processing regions, dorsolateral prefrontal cortex and parietal cortex control regions 관련,\\
비단 MTL은 아냐\\

\paragraph{working memory와 imagery의 차이}
\begin{itemize}
  \item working memory literature: delay period 동안 자극이 `유지'된다.
  \item imagery literature: delay period 동안 짜극이 `상상'된다.
\end{itemize}
$\longrightarrow$ 만약 미래에 이 둘이 서로 다른 뇌 영역과 관련있을 보이면 둘이 다름을 말할 수 있지만, 책 왈, 할 수 있겠니??(This seems unlikely)

\paragraph{long-term memory와 imagery의 *이}
\begin{itemize}
  \item long-term memory: 첫 encoding 이후 mind에 없던 정보를 retrieval해야 한다.
  \item visual imagery: mind에 유지해야 한다.
\end{itemize}
$\longrihgtarrow$ MTL(hippocampus를 포함한): LTM ㅇㅇ, imagery ㄴㄴ\\

\tbox{fMRI 실험}{
  visual recollection and visual imagery의 공통 관련된 뇌 영역과 구별된 뇌 영역 구하고파.\footnote{\url{https://kosslynlab.fas.harvard.edu/sites/g/files/omnuum10906/files/kosslynlab/files/slotnick\_et\_al\_cogn\_neurosci\_2012.pdf}}

  \begin{center}
    \includegraphics[width=0.6\textwidth]{image/ch8_bp6}
    \captionof{figure}{(a) During familiarization, pictures of objects were memorized, imagined, and then the mental image was compared to the picture of the
object and corrected (words in parentheses were not shown during the experiment). (b) During both memory and imagery study phases, pictures of
objects from familiarization were presented. During the memory-test phase, words corresponding to old (studied) objects, new words, and control
words were presented at fixation. Participants responded “remember” (R, 1 key), “know” (K, 2 key), or “new” (N, 3 key) to old-item words and new-
item words, and responded with the corresponding response key to control words (“left,” “center,” or “right”). Example responses and the
corresponding event types are shown to the right of each word. (c) During the imagery-test phase, the same word types were presented and participants
responded “high vividness” (H, 1 key), “moderate vividness” (M, 2 key), or “low vividness” (L, 3 key) to old-item words and new-item words in
addition to the same control responses. Example responses and the corresponding event types are shown to the right of each word.}
    \label{fig:ch8_bp6}
  \end{center}

  \paragraph{청사진}
  \begin{itemize}
    \item familiarization phase:\\
      zebra and a feather 등 물체로 이루어진 list가 존재한다. 예컨대 처음은 얼룩말이다.
      피험자는 얼룩말을 본다. 피험자가 button을 누른다. 얼룩말이 사라졌다. 피험자는 얼룩말을 상상한다.
      피험자가 button을 다시 누른다. 다시 얼룩말이 나온다. 다음 물체로 넘어간다. 해당 list는 3번 반복한다.\\
      $\Rightarrow$ imagery studies서 많이 쓰이는 방법이다.
    \item study phase:\\
      각 항목을 기억하라고 지시한다. claude 왈, long-term memory를 위한 과정이다.
    \item memory test phase:\\
      old objects, new objects, or control responses (`left', `center', or `right')가 있다.
      old가 나오면 피험자는 ``remember''\footnote{디테일해부려}–``know''–``new''를 판다, new가 나오면 new, control responses가 나오면 지시에 따라 left면 left 버튼 눌러야 한다.\footnote{claude 왈, control words는 순수한 motor response 측정,
Memory/Imagery processing과 구분,
Baseline 활성화 설정을 위해 필요하다. 이후 책: which required word processing and motor processing
}
    \item imagery test phase:\\
      ``High vividness'' - ``Medium vividness'' - ``Low vividness''를 판단한다.
  \end{itemize}

  \paragraph{결과}
  memory-old-``remember''와 imagery-old-``high vividness''의 반응은 control responses와 비교할 때 dorsolateral prefrontal cortex, the parietal cortex, and visual sensory regions, including V1를 활성화한다.\\
  memory-old-``remember''와 imagery-old-``high vividness''의 반응을 비교하면 visual sensory regions는 memory 과제와 더 큰 관련이 있다.\\
  $\longrightarrow$ long-term memory와 imagery는 많은 영역을 공유한다. 비단, 동일한 인지 가정 아냐
  
  \begin{center}
    \includegraphics[width=0.6\textwidth]{image/ch8_result3}
    \captionof{figure}{Common activity associated with memory-old-
remember and imagery-old-high vividness projected onto an inflated
cortical surface representation (gyri and sulci are shown in light and
dark gray, respectively; top, superior view; bottom, posterior-inferior
view; key at center).
Differential activity associated with memory-old-
remember and imagery-old-high vividness projected onto an inflated
cortical surface representation (key at center).}
    \label{fig:ch8_result3}
  \end{center}
}

\paragraph{autobiogrphical memory와 imagined autobiographical memory}
\begin{itemize}
  \item autobiographical memory:\\
    과거 사건과 관련되어 있다.
  \item imagined autobiographical memory:\\
    미래 사건과 관련되어 있다.
\end{itemize}

fMRI 실험을 하였다.\\
visual regions, the dorsolateral prefrontal cortex, the parietal cortex,
and the medial temporal lobe (including the hippocampus)가 활성화 된다.\\
$\longrightarrow$ imagery가 MTL과 관련되어 보일 수 있다. 비단!! 상상하면서 과거 정보를 가져오고 이 과정서 MTL이 활성화 된 것일 수도 있다.


\section{Language and Memory\index{language}}
\label{sec:language}
환자가 있다.
\begin{itemize}
  \item lesion in the left inferior dorsolateral prefrontal cortex $\longrightarrow$ isolated word production deficit\\
    말할 수 없지만 말을 이해할 수 있다. $\longrightarrow$ Broca’s area
  \item lesion in the left posterior superior temporal cortex $\longrightarrow$ isolated comprehension deficit\\
    이해할 수 없지만 말할 수 있다. $\longrightarrow$ Wernicke’s area
\end{itemize}

\begin{figure}[htbp]
  \centering
  \includegraphics[width=0.6\textwidth]{image/ch8_language}
  \caption{Language processing regions (lateral view, occipital pole to the right).
Regions (in different shades of gray) are labeled and arrows indicate the direction of
information flow between regions. + angular gyrus = BA39, inferior parietal cortex}
  \label{fig:ch8_language}
\end{figure}

language processing과 관련된 영역이에요.

\paragraph{the classic model of language processing}
\begin{table}[htb]
	\centering
	\begin{tabular}{|c|c|c|}
		\hline
    \thead{기능} & \thead{관련된 것} & \thead{위치} \\
		\hline
    Word production & Broca’s area & inferior and anterior to the motor cortex \\
    visual word comprehension\footnote{시각적인 단어 이해??, 긁 읽기} & visual cortex, angular gyrus, Wernicke’s area \\
		\hline
	\end{tabular}
	\caption{각 브로드만 영역의 매핑}
\end{table}
Broca’s area $\longrightarrow$ 언어 생성\\
Wernicke’s area $\longrightarrow$ 언어 이해

\paragraph{최근 증거}
언어 생성과 이해는 Broca's area와 Wernicke's area와 관련되어 있다.\\
단어 뜻 처리(언어 분야서 semantic processing라고 부름요)는 Broca’s area, Wernicke’s area, the angular gyrus, and more anterior
superior temporal cortex를 활성화한다.\\
중요한 점: semantic/conceptual processing는 the left inferior dorsolateral prefrontal
cortex and the left posterior superior temporal cortex를 활성화 한다;;

\paragraph{언어와 memory}
언어 처리(특히 단어 처리)는 memory studies에서 중요한 면이다:
단어를 짜극으로 주는 경우 허다하고, 의미있는 물체는 semantic processing과 관련있다.\\
에컨대 양을 보여 주었다. 떠올린다, 무엇을?:
\begin{itemize}
  \item 소리: `bah'
  \item 위치: 농장
  \item 사람에게 어떻게 유용한지(꺄 잔인쓰;;): 양모
\end{itemize}
$\Longrightarrow$ semantic processing $\rightarrow$ language processing regions(the left inferior dorsolateral prefrontal cortex and
the left posterior superior temporal cortex (i.e., Broca’s area and
Wernicke’s area, respectively))

memory 연구서 language processing 사례:
\begin{itemize}
  \item Semantic memory: the left dorsolateral prefrontal cortex (Ch 3)
  \item False memory: the left dorsolateral prefrontal cortex and the left posterior superior
temporal cortex (Ch 5)
  \item Conceptual priming: the left dorsolateral prefrontal cortex and the
left posterior superior temporal cortex (Ch 7)
\end{itemize}

+ left DL-PFC가 활성화 되었다고 해서 꼭 언어처리인 것은 아니다.


\section{Emotion and Memory\index{emotion}}
\label{sec:emotion}
정서 신경과학(affective neuroscience)은 emotional processing과 관련된 뇌 영역을 다루며 인지 신경과학과 다르다.
그런데! 인지 신경과학서 감정(e.g., fear, disgust, or happiness)을 유발하는 자극을 사용할 때 중복되는 부분이 생긴다.\\
거미, 해골, 총(감정 유발! $\rightarrow$ neutral stimuli가 아님) $\longrightarrow$ the amygdala(편도체), the orbitofrontal cortex (눈/안구 바로 앞 frontal cortex), and the dorsolateral prefrontal cortex 활성화

\begin{figure}[htbp]
  \centering
  \includegraphics[width=0.6\textwidth]{image/ch8_amygdala}
  \caption{The amygdala and the hippocampus. The amygdala (in dark gray) and the
hippocampus (in light gray) in each hemisphere are shown within a semi-transparent brain
(lateral-anterior view, occipital pole to the right).}
  \label{fig:ch8_amygdala}
\end{figure}

편도체이다. 감정과 관련되어 중요오오오하고 뇌 많은 부분과 연결되어 hub 역할 하는 것 같다고 한다.\\

+ DL-PFC가 활성화 되었다고 해서 하나아ㅢ 인지 과정이 발생했다고 생각하면 아니된다. 책에서 DL-PFC가 하는 역할에 대한 2가지 가설(1. inhibition \& Selection 2. flexible region that reorganizes its
function)을 제시하는데 궁금하면 Box 8.2 ㄱㄱ\\

+ 감정과 관련된 것이 더 뇌에서 학습되고 강화된 부분과 관련되어 있는 것 같다고 한다.

\section{quiz}
\begin{itemize}
  \item
Which brain regions have been associated with visual attention and visual working memory?
  \item
How do the brain regions associated with visual attention and visual long-term memory differ?
  \item
Are imagery and working memory different cognitive processes?
  \item
What are the two primary brain regions associated with language
processing?
  \item
Which brain region interacts with the hippocampus during memory for emotional information?
\end{itemize}

\end{document}
