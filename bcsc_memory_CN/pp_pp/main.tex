\documentclass{beamer}
% Theme choice:
\usetheme{CambridgeUS}
\usecolortheme{dolphin}
\usepackage{kotex}
\usepackage{moreenum}
\usepackage{amsmath}
\usepackage{polynom}
\usepackage{enumitem}
\usepackage{graphicx}
\usepackage{tabularx}
\usepackage{subfig}
\usepackage{enumitem}
\usepackage{siunitx}
\usepackage{subcaption}
\usepackage{multirow}
\usepackage{array}

\setitemize{label=\usebeamerfont*{itemize item}%
    \usebeamercolor[fg]{itemize item}
    \usebeamertemplate{itemize item}}

\usepackage{scrextend}
\changefontsizes{8pt}

\usepackage[font=tiny]{caption} % ,labelfont=bf]{caption}
\setbeamertemplate{caption}[numbered]

\setbeamercolor{block title}{bg=blue!15}
\setbeamercolor{block body}{bg=blue!5}

\usepackage{fontawesome}

\usepackage{caption}
\captionsetup{font=normalsize}
\usepackage{wrapfig}


% \newcommand\blfootnote[1]{%
% \begingroup
% \renewcommand\thefootnote{}%
% \footnote{\footnotesize#1}%
% \addtocounter{footnote}{-1}%
% \endgroup
% }
% 
% \renewcommand{\footnotesize}{\footnotesize}

% \usecolortheme{beaver}
% Title page details: 

\PassOptionsToPackage{demo}{graphicx}
\makeatletter
\newcommand\titlegraphicii[1]{\def\inserttitlegraphicii{#1}}
\titlegraphicii{}
%\setbeamercolor{title}{bg=blue!20}

\setbeamertemplate{title page}
{
  \vbox{}
  {\usebeamercolor[fg]{titlegraphic}\inserttitlegraphic\hfill\inserttitlegraphicii\par}
  \begin{centering}
    \begin{beamercolorbox}[sep=8pt,center,rounded=true,shadow=true]{title}
      \Huge{\usebeamerfont{title}}\inserttitle\par%
      \ifx\insertsubtitle\@empty%
      \else%
        \vskip0.25em%
        {\large{\usebeamerfont{subtitle}}\usebeamercolor[fg]{subtitle}\insertsubtitle\par}%
      \fi%     
    \end{beamercolorbox}%
    \vskip1em\par
    \begin{beamercolorbox}[sep=8pt,center]{date}
      \usebeamerfont{date}\insertdate
    \end{beamercolorbox}%\vskip0.5em
    \begin{beamercolorbox}[sep=8pt,center]{author}
      \usebeamerfont{author}\insertauthor
    \end{beamercolorbox}
    \begin{beamercolorbox}[sep=8pt,center]{institute}
      \usebeamerfont{institute}\insertinstitute
    \end{beamercolorbox}
  \end{centering}
  %\vfill
}
\makeatother

\usepackage{datetime}
\newdate{date}{13}{09}{2025}
\date{\displaydate{date}}

\title{Science of Memory}
\subtitle{심화스터디}
%\subtitle{Chapter 6. Visualizing Nervous System Structure\\신경계 구조 시각화}
\author{김태현\and 윤시원\and 정유경}
\institute{BCSC 2025}
\date{\displaydate{date}}
\usepackage{tikz}
% \titlegraphic{\includegraphics[width=2cm]{image/bcsc_logo2}}
\titlegraphic { 
\begin{tikzpicture}[overlay,remember picture]
\node[right=0.2cm] at (current page.150){
    \includegraphics[width=2cm]{image/bcsc_logo2}
};
\end{tikzpicture}
}

\begin{document}
\begin{frame}
    \titlepage
\end{frame}

\begin{frame}{Contents}
  \tableofcontents
\end{frame}

\section{스터디 소개}
\begin{frame}{Science of Memory 스터디 소개}
  \begin{columns}
    \begin{column}{0.39\textwidth}
      \subsection{스터디 목표}
      \begin{block}{스터디 목표}
        기억의 기본 원리부터 최신 연구 동향, 다양한 응용 분야까지 깊이 있게 탐구
      \end{block}
      \subsection{스터디 진행 방식}
      \begin{block}{스터디 진행 방식}
        매주 1회의 대면 활동을 기본으로 진행
      \end{block}
      \vspace{-1em}
      \begin{figure}
        \centering
        \subfloat[{Learning and memory}]
          {\includegraphics[height=2.5cm]
          {image/LN}}\qquad
        \subfloat[{Cognitive Neuroscience of Memory}]
          {\includegraphics[height=2.5cm]
          {image/CN}}
        \vspace{-0.5em}
        \caption{스터디 교재}
      \end{figure}
      % \begin{figure}
      %   \includegraphics[width=\textwidth]{image/ch2_blueprint}
      %   \caption{Dishbrain arichitecture\footnote{Kagan, Brett J. et al. 2022}}
      % \end{figure}
    \end{column}
    \hfill
    \begin{column}{0.6\textwidth}
      % \vspace{-1em}
      \subsection{스터디 일정}
      \begin{table}[]
        \centering
        % \tiny
        \scriptsize
        \begin{tabularx}{\textwidth}{|c|p{1cm}|X|l|}
      \hline
      \textbf{차시}        & \textbf{materials}                                            & \textbf{발제 내용}                                                        & \textbf{발제자} \\ \hline
      \multirow{4}{*}{1} & \multirow{4}{*}{\parbox[c]{1cm}{Learning and Memory}}                 & 1.01 Learning Theory and Behavior: Introduction and Overview & 공동           \\ \cline{3-4} 
                         &                                                      & 1.02 A Typology of Memory Terms                              & 윤시원          \\ \cline{3-4} 
                         &                                                      & 1.03 Retrieval From Memory                                   & 김태현          \\ \cline{3-4} 
                         &                                                      & 1.15 Memory for Space, Time, and Episodes                    & 정유경          \\ \hline
      \multirow{2}{*}{2} & \multirow{11}{*}{\parbox[c]{1.1cm}{Cognitive Neuroscience of Memories}} & 1. Types of Memory and Brain Regions of Interest             & 윤시원          \\ \cline{3-4} 
                         &                                                      & 2. The Tools of Cognitive Neuroscience                       & 정유경          \\ \cline{1-1} \cline{3-4} 
      \multirow{3}{*}{3} &                                                      & 3. Brain Regions Associated with Long-Term Memory            & 윤시원          \\ \cline{3-4} 
                         &                                                      & 4. Brain Timing Associated with Long-Term Memory             & 공동           \\ \cline{3-4} 
                         &                                                      & 5. Long-Term Memory Failure                                  & 정유경          \\ \cline{1-1} \cline{3-4} 
      \multirow{2}{*}{4} &                                                      & 6. Working Memory                                            & 윤시원          \\ \cline{3-4} 
                         &                                                      & 7. Implicit Memory                                           & 정유경          \\ \cline{1-1} \cline{3-4} 
      \multirow{2}{*}{5} &                                                      & 8. Memory and Other Cognitive Processes                      & 윤시원          \\ \cline{3-4} 
                         &                                                      & 9. Explicit Memory and Disease                               & 김태현          \\ \cline{1-1} \cline{3-4} 
      \multirow{2}{*}{6} &                                                      & 10. Long-Term Memory in Animals                              & 정유경          \\ \cline{3-4} 
                         &                                                      & 11. The Future of Memory Research                            & 김태현          \\ \hline
      \multirow{3}{*}{7} & \multirow{3}{*}{논문}                                  & Engram                                                       & 김태현          \\ \cline{3-4} 
                         &                                                      & Memory systems in AI                                         & 윤시원          \\ \cline{3-4} 
                         &                                                      & Multimodal Brain Imaging in Memory Research                  & 정유경          \\ \hline
      \end{tabularx}
      \caption{Science of Memory 스터디 일정표}
      \end{table}

    \end{column}
  \end{columns}
\end{frame}

\section{스터디 내용 소개}
\subsection{기억의 종류}
\begin{frame}{기억의 종류}
  % {\huge{\textbf{기억의 종류}}}\\
  \begin{table}
    \begin{tabular}{l}
      Consciousness?\\
      Kept in mind during `delay period'?\\
      Involve previous episode?\\
      Task performance:\\
      Subjective experience:\\
      General term:
    \end{tabular}
  \end{table}

  % \begin{tikzpicture}[remember picture, overlay]
  %   \node (image) 
  %   [
  %     left=0.2cm
  %   ] at (current page.east)
  %   {
  %     \includegraphics[width=0.8\textwidth]{image/memory_tree}
  %   };
  %   \node
  %   [
  %       fill=teal,
  %       align=center,
  %       text=white,
  %       font={\huge\bfseries}
  %   ] at (image.center) {A beautiful photo!!};
  % \end{tikzpicture}
\end{frame}

\subsection{장기 기억}
\begin{frame}{장기 기억\textsuperscript{Long-Term Memory}\\\normalsize{: 오랜 시간 동안 정보를 저장하는 시스템}}
  \noindent
  \begin{minipage}{0.4\textwidth}
    \vspace{2em}
    \textbf{실험 패러다임}:\\
    Study phase \\$\rightarrow$ Delay period(수\textit{분}$\sim$수\textit{시간}) \\$\rightarrow$ Test phase(Old-new recognition)\\~\\~\\~\\
  \end{minipage}%
  \begin{minipage}{0.6\textwidth}
      \centering
      \begin{figure}
      \subfloat[Study]
        {\includegraphics[height=1.5cm]
        {image/study_phase}}\qquad
      Delay...
      \subfloat[Test]
        {\includegraphics[height=1.5cm]
        {image/test_phase}}
      \vspace{-0.5em}
      \caption{실험 패러다임}
    \end{figure}
  \end{minipage}\\
  \vspace{-2em}
  % \begin{wrapfigure}{r}{0.37\textwidth}
  %   \centering
  %   \subfloat[Study]
  %     {\includegraphics[height=1.5cm]
  %     {image/study_phase}}\qquad
  %   Delay...
  %   \subfloat[Test]
  %     {\includegraphics[height=1.5cm]
  %     {image/test_phase}}
  %   \vspace{-0.5em}
  %   \caption{스터디 교재}
  % \end{wrapfigure}
  % \vspace{-2em}
  % \qquad\\
  % \vspace{2em}
  % \textbf{실험 패러다임}:\\
  % Study phase \\$\rightarrow$ Delay period(수\textit{분}$\sim$수\textit{시간}) \\$\rightarrow$ Test phase(Old-new recognition)\\~\\~\\~\\
  \textbf{Retrieval 시 활성화 뇌 영역}:\\
  감각 영역, 통제 영역(Medial Temporal Lobe\textsuperscript{MTL}, \\\qquad dorsolateral Prefrontal Cortex\textsuperscript{dl-PFC}, Parietal Cortex)
  \vspace{-3.5em}
  \newsavebox{\rightfigure}
\savebox{\rightfigure}{%
    \subfloat[Model of MTL sub-region function]
    {\includegraphics[height=3.2cm]{image/hippocampus_binding}}%
}
  \begin{figure}
    \centering
    \only<1-5>{\vspace{-1em}}
    \subfloat[Regions of the brain associated with episodic memory]
      {
        \begin{tikzpicture}
          \node[anchor=south west,inner sep=0] (image) at (0,0) {
            \includegraphics[height=2cm]{image/brain_regions_1}%
          }
          \draw<2->[red,very thick] (4.55,0.35) circle (0.29cm);
          \draw<3->[red,very thick] (0.9,1.3) circle (0.29cm);
          \draw<4->[red,very thick] (2.2,1.1) circle (0.29cm);
          \draw<5->[red,very thick] (4.3,1.47) circle (0.29cm);
        \end{tikzpicture}
        % \includegraphics<1>[height=2cm]{image/brain_regions_1}%
        % \includegraphics<2>[height=2cm]{image/brain_regions_2}%
        % \includegraphics<3>[height=2cm]{image/brain_regions_3}%
        % \includegraphics<4>[height=2cm]{image/brain_regions_4}%
        % \includegraphics<5->[height=2cm]{image/brain_regions_5}%
      }\qquad
    \only<1-5>{\phantom{%
      \subfloat[Model of MTL sub-region function]
      {\includegraphics[height=3.2cm]{image/hippocampus_binding}}%
    }}
    \only<6->{
      \subfloat[Model of MTL sub-region function]
      {\includegraphics[height=3.2cm]{image/hippocampus_binding}}
    }
    \vspace{-0.5em}
    \caption{Retrieval 시 활성화되는 뇌 영역과 해마의 역할}
  \end{figure}
  % \begin{figure}
  %   \centering
  %   \subfloat[Regions of the brain associated with episodic memory]
  %     {
  %       \only<1>{\includegraphics[height=2cm]{image/brain_regions_1}}
  %       \only<2>{\includegraphics[height=2cm]{image/brain_regions_2}}
  %       \only<3>{\includegraphics[height=2cm]{image/brain_regions_3}}
  %       \only<4>{\includegraphics[height=2cm]{image/brain_regions_4}}
  %       \only<5->{\includegraphics[height=2cm]{image/brain_regions_5}}
  %     }\qquad
  %   \only<1-5>{\phantom{%
  %       \subfloat[Model of MTL sub-region function]
  %       {\includegraphics[height=3.2cm]{image/hippocampus_binding}}%
  %   }}
  %   \only<6->{
  %     \subfloat[Model of MTL sub-region function]
  %     {\includegraphics[height=3.2cm]{image/hippocampus_binding}}
  %   }
  %   \vspace{-0.5em}
  %   \caption{Retrieval 시 활성화되는 뇌 영역과 해마의 역할}
  % \end{figure}

\end{frame}


\begin{frame}
  \\~\\
  \textbf{공고화(Consolidation)}: 새로운 기억을 안정적인 장기 기억으로 전환하는 과정
  \begin{itemize}
    \small
    \item \textbf{표준 통합 이론(Standard consolidation model)}: hippocampal-cortical $\rightarrow$ cortical-cortical interaction
    \item \textbf{다중 흔적 이론(Multiple trace theory)}: hippocampus is involved in LTM throughout the lifetime
  \end{itemize}
  \vspace{1.5em}
  \textbf{부호화(Encoding)}: 정보의 습득
  \begin{minipage}{0.7\textwidth}
    \vspace{-1em}
    \begin{itemize}
      \item Subsequent memory analysis로 관련 뇌 영역 규명
      \item Medial Temporal Lobe, dorsolateral Prefrontal Cortex, Parietal Cortex $\rightarrow$ Retrieval과 차이 有
    \end{itemize}
    \hfill
  \end{minipage}%
  \begin{minipage}{0.3\textwidth}
    \vspace{-2em}
    \centering
    \includegraphics[height=1.5cm]{image/encoding_brain_regions}
    \captionof{figure}{\small Regions of the brain associated with subsequent memory effects}
  \end{minipage}

  %\vspace{-1em}
  \textbf{망각(Forgetting)}:
  \begin{minipage}{0.5\textwidth}
    \begin{itemize}
      \small
      \item \textbf{Typical forgetting}: attention 부족으로 encoding 실패. dlPFC, mPFC, parietal cortex 활성화. DMN과 같은 패턴의 뇌 활성
      \item \textbf{Retrieval-induced forgetting}: 관련 기억의 방해를 받아 회상 실패 (fMRI 실험)
      \item \textbf{Motivated forgetting}: 의도적인 망각
    \end{itemize}
    \hfill
  \end{minipage}%
  \begin{minipage}{0.5\textwidth}
    \vspace{-2em}
    \begin{figure}
      \centering
      \subfloat[\small Subsequent forgetting fMRI activity and default network fMRI activity]
        {
          \includegraphics[height=2cm]{image/forgetting_brain_regions}
        }\qquad
      \subfloat[\small Retrieval-induced forgetting EEG activity]
        {\includegraphics[height=1.9cm]
        {image/forgetting_graph}}
      \vspace{-0.5em}
      \caption{Forgetting}

    \end{figure}
  \end{minipage}
  
\end{frame}

\subsubsection{Retrieval-induced forgetting fMRI 실험}
\begin{frame}{Retrieval-induced forgetting fMRI 실험}
  \begin{columns}
    \begin{column}{0.38\textwidth}
      \begin{figure}
        \includegraphics[width=\textwidth]{image/Retrieval-induced_forgetting}
        \caption{Retrieval-inducted forgetting paradigm, behavioral performance, and fMRI activity}
      \end{figure}
    \end{column}
    \hfill
    \begin{column}{0.6\textwidth}
      % \large
      \textbf{Used Paradigm}:
      \vspace{-0.5em}
      \begin{itemize}[itemsep=0pt, parsep=0pt]
        \small
        \item \textbf{initial study phase}: word pairs(category+examplar) presented
        \item \textbf{intermediate retrieval practice phase}: subset of categories(category+two-letter word cue)\\
        $\longrightarrow$ non-presented words from this category are inhibited
        \item \textbf{final recall phase}: word pairs(category+examplar) presented
      \end{itemize}
      \vspace{0.5em}
      \textbf{Examplar 분류}:
      \vspace{-0.5em}
      \begin{itemize}[itemsep=0pt, parsep=0pt]
        \small
        \item 대조군(non-practiced categories): low freq. (\textbf{C+}), high freq. (\textbf{C-})
        \item 실험군(practiced categories): low freq. but practiced (\textbf{P+}), high freq. but not practiced (\textbf{P-})
      \end{itemize}
      \vspace{0.5em}
      \textbf{Result}:
      \vspace{-0.5em}
      \begin{itemize}[itemsep=0pt, parsep=0pt]
        \small
        \item P-는 P+에 의해 억제되어 대조군보다 낮은 연상 빈도를 보임
        \item P+은 연습의 결과로 대조군보다 높은 연상 빈도를 보임
        \item comparison from P- to P+: \textbf{dlPFC}의 활성이 클수록, \textbf{RIF 효과}가 더 높은 것으로 나타남
      \end{itemize}

      % \textbf{Retrieving particular information} from memory facilitates the later retrieval of that information, but also \textbf{impairs the later retrieval of related, interfering information}.\\~\\

      % Activity in \textbf{left anterior VLPFC (BA 47)} and left posterior temporal cortex (BA 22), regions implicated in the controlled \textbf{retrieval of weak semantic memory representations}, predicted the degree of retrieval-induced forgetting.
    \end{column}
  \end{columns}
\end{frame}

\subsubsection{관련 ERP}
\begin{frame}{관련 ERP}
  \begin{columns}
    \begin{column}{0.5\textwidth}
      \begin{block}{Familiarity \& Recollection}
        \textbf{Familiarity} (Know)
        \begin{itemize}
          \item mid-frontal old-new effect
          \item frontal brain activity within \textbf{300$\sim$500ms}
          \item FN400: negative frontal activity in 400ms
        \end{itemize}\\~\\~\\
        \textbf{Recollection} (Remember)
        \begin{itemize}
          \item left-parietal old-new effect
          \item parietal brain activity within \textbf{500$\sim$800ms}
        \end{itemize}
      \end{block}
    \end{column}
    \hfill
    \begin{column}{0.48\textwidth}
      \begin{figure}
        \centering
        \includegraphics[width=\textwidth]{image/familiarity_ERP}
        \caption{ERP activity associated with recollection and familiarity}
      \end{figure}
    \end{column}
  \end{columns}
\end{frame}

\subsubsection{관련 Frequency Band}
\begin{frame}{관련 Frequency Band}
  \begin{columns}
    \begin{column}{0.58\textwidth}
      \begin{block}{Frequency band}
        \begin{itemize}
          \item \textbf{gamma}($>$30Hz): 서로 다른 피질 영역의 기능 처리 
          \item \textbf{alpha}(8$\sim$12Hz): 피질 영역 활동 억제
          \item \textbf{theta}(4$\sim$8Hz): 해마와 피질 영역 간 상호작용
        \end{itemize}
      \end{block}
      \begin{block}{LTM encoding \& retrieval}
        \begin{itemize}
          \item gamma increase in \textbf{parietal}, \textbf{occipital} regions
          \item alpha decrease in \textbf{anterior}, \textbf{posterior} regions
          \item theta increase in \textbf{frontal} regions, \textbf{thalamus}
          \item cross-frequency coupling\footnote{두 영역 간 상호작용} \\b/w \textbf{frontal} theta \& \textbf{parietal-occipital} gamma,\\b/w \textbf{frontal} \& \textbf{thalamic} theta
        \end{itemize}
      \end{block}
    \end{column}
    \hfill
    \begin{column}{0.38\textwidth}
      \begin{figure}
        \centering
        \includegraphics[width=0.8\textwidth]{image/remember_EEG}
        \caption{EEG frequency band activity associated with subsequently remembered and forgotten items}
      \end{figure}
    \end{column}
  \end{columns}
\end{frame}

\subsubsection{수면 중 역할}
\begin{frame}{수면 중 역할}
  \begin{columns}
    \begin{column}{0.58\textwidth}
      \textbf{서파 수면}(slow-wave sleep)
      \begin{itemize}
        \item 장기 기억 공고화가 주로 발생하는 시점
        \item 1Hz 이하 주파수에서의 widespread cortical modulation 수반
        \item 주로 3 or 4단계 REM 수면, 낮잠 중 발생
        \item 다른 뇌파(11$\sim$16Hz에서 진동하는 thalamic-cortical sleep spindles, 200Hz 근처에서 진동하는 hippocampal sharp-wave ripples\footnote{hippocampal-cortical 상호작용을 조절하여 기억(from the previous waking period)의 replay를 반영})와 synchronize되어 기억 공고화에 최적화
      \end{itemize}
    \end{column}
    \hfill
    \begin{column}{0.42\textwidth}
      \begin{figure}
        \centering
        \includegraphics[width=\textwidth]{image/sleep_stages}
        \caption{Sleep stages and brain oscillations associated with slow wave sleep and long-term memory consolidation}
      \end{figure}
    \end{column}
  \end{columns}
\end{frame}

\subsection{작업 기억}
\begin{frame}{작업 기억\textsuperscript{Working Memory}\\\normalsize : 짧은 시간 동안 정보를 유지(maintenance)하고, 그 정보를 조작/이용해 현재 과제를 수행하는 인지 시스템
}
  \begin{block}{Working Memory vs Short-Term Memory (단기기억, STM)}
    \begin{itemize}
      \item 두 용어는 종종 혼용되지만 (본 교재), 현대 인지신경과학에서는 개념적으로 구분하는 경향이다.
      \item 초기에는 단기 기억으로 불렸으나, 정보의 `능동적 처리' 기능이 강조되면서 작동 기억이라는 용어가 등장했다.
      \item \textbf{Short-Term Memory}: \\정보를 짧은 시간동안 단순히 유지하는 것 (수동적)
      \item \textbf{Working Memory}: \\정보를 짧은 시간동안 유지할 뿐만 아니라, 그 정보를 \textbf{능동적으로 조작하고 활용}하는데 쓰인다.\\
        $\rightarrow$ 단기 기억의 저장 특징 + \textbf{저장된 정보를 활용(조작/처리)}, 더 포괄적
    \end{itemize}
  \end{block}
\end{frame}

\subsubsection{Baddeley's Model}
\begin{frame}{Baddeley's Model of Working Memory}
  \begin{columns}
    \begin{column}{0.28\textwidth}
      \centering
      \begin{figure}
        \centering
        \includegraphics[width=\textwidth]{image/baddeley_model}
        \caption{Baddeley’s working memory model}
      \end{figure}
    \end{column}
    \hfill
    \begin{column}{0.72\textwidth}
      \vspace{-2em}
      \large
      \begin{itemize}
        \item \textbf{Central executive}: \\주의력 배분, 하위 시스템의 작업 조율 \& 총괄
        \item \textbf{Visuospatial sketchpad}: \\시각/공간 정보를 유지하고 조작
        \item \textbf{Phonological loop}: \\언어/청각 정보를 일시적 저장 및 되뇔 때 관여
        \item \textbf{Episodic buffer}: \\서로 다른 종류의 감각 정보들을 하나의 일화로 통합하고, 이 정보를 장기 기억으로 형성하거나 장기 기억 속의 일화를 인출할 때 사용되는 임시 작업 공간
      \end{itemize}
    \end{column}
  \end{columns}
\end{frame}

\subsubsection{관련 뇌 영역}
\begin{frame}{Brain Regions Related to Working Memory}
  \begin{columns}
    \begin{column}{0.31\textwidth}
      \begin{block}{dlPFC}
        \begin{minipage}[t][5cm][c]{\textwidth}
          \centering
          \includegraphics[width=\textwidth]{image/working_memory_b1}
          \textbf{Control Tower}\\
          어떤 정보에 주의를 기울이고, 어떻게 처리할지 지시하는 상위 제어 기능을 수행
        \end{minipage}
      \end{block}
    \end{column}
    \hfill
    \begin{column}{0.31\textwidth}
      \begin{block}{Parietal Cortex}
        \begin{minipage}[t][5cm][c]{\textwidth}
          \centering
          \includegraphics[width=\textwidth]{image/working_memory_b2}\\~\\~\\
          주로 주의 집중과 정보의 공간적 위치를 파악하는 데 관여
        \end{minipage}
      \end{block}
    \end{column}
    \hfill
    \begin{column}{0.31\textwidth}
      \begin{block}{Sensory Processing Regions}
        \begin{minipage}[t][5cm][c]{\textwidth}
          \centering
          \includegraphics[width=\textwidth]{image/working_memory_b3}\\~\\~\\
          작업 기억의 내용이 저장됨
        \end{minipage}
      \end{block}
    \end{column}
  \end{columns}\\~\\~\\
  \large ** 기본적이고 단순한 작업기억은 대체적으로 해마에 크게 의존하지 않음
\end{frame}

\subsubsection{관련 Frequency Band}
\begin{frame}{Brain Activity Related to Working Memory}
  \begin{columns}
    \begin{column}{0.52\textwidth}
      \large
      \textbf{Gamma Activity} ($>$30 Hz):\\여러 뇌 영역에 흩어져 있는 정보 조각들을 하나의 의미 있는 기억으로 묶어주는 `정보 통합(Binding)' 역할
    \end{column}
    \hfill
    \begin{column}{0.48\textwidth}
      \centering
      \begin{figure}
        \centering
        \includegraphics[width=\textwidth]{image/gamma}
        \caption{Gamma Activity}
      \end{figure}
    \end{column}
  \end{columns}
  \begin{columns}
    \begin{column}{0.52\textwidth}
      \large
      \textbf{Alpha Activity} (8$\sim$12 Hz):\\불필요한 시각 정보 처리를 줄이기 위해 과제와 관련 없는 뇌 영역의 활동을 억제하는 역할
    \end{column}
    \hfill
    \begin{column}{0.48\textwidth}
      \centering
      \begin{figure}
        \centering
        \includegraphics[width=\textwidth]{image/alpha}
        \caption{Alpha Activity}
      \end{figure}
    \end{column}
  \end{columns}
\end{frame}

\subsubsection{실험 패러다임}
\begin{frame}{실험 패러다임}
  정보를 \textbf{(1) 활성화된 상태로 유지}하고 \textbf{(2) 처리}하는 능력을 측정함\\
  대부분의 패러다임은 공통적으로 [학습 $\rightarrow$ 지연 $\rightarrow$ 검사]의 3단계 구조를 따름\\~\\~\\

  \textbf{실험 패러다임}:
  \begin{itemize}
    \item \textbf{학습 단계 (Study Phase)}: 참가자에게 기억해야 할 정보를 제시
    \item \textbf{지연 기간 (Delay Period)}: 몇 초 $\sim$ 몇십 초 동안 아무것도 안 보여줌\\
      $\rightarrow$ 참가자는 이 시간 동안 \textbf{학습한 정보를 마음속으로 적극적으로 유지}해야 함(작업 기억의 핵심적인 측정 구간)
    \item \textbf{검사 단계 (Test Phase)}: 학습한 정보에 대해 질문하여 정확도를 측정함
  \end{itemize}
\end{frame}

\begin{frame}{작업 기억 실험의 대표적인 패러다임들}
  \begin{columns}
    \begin{column}{0.5\textwidth}
      \begin{figure}
        \centering
        \includegraphics[width=0.8\textwidth]{image/work_ex_p1}
        \caption{작업 기억 실험 패러다임 예시}
      \end{figure}
      \textbf{Old-New Recognition}:\\
      \vspace{-0.5em}
      \begin{itemize}
        \small
        \item \textbf{학습 단계}: 여러 항목(단어, 도형 등)을 보여줌
        \item \textbf{검사 단계}: 학습 단계 항목과 새로운 항목을 섞어서 제시 \\$\Rightarrow$ 참가자는 각 항목이 이전에 봤던 것(old)인지 처음 보는 것 (new)인지 판단
      \end{itemize}\\~\\
      
      \textbf{Source/Context Memory Task}:\\
      {\small Old-new recognition + 그 항목이 제시된 위치나 색상 등의 맥락까지 기억}
    \end{column}
    \begin{column}{0.5\textwidth}
      \textbf{N-Back Task}:\\
      {\small 일련의 자극(예: 자음 문자열)을 듣거나 보면서, \textbf{현재 자극이 N개 전에 제시된 자극과 일치하는지 여부}를 판단}\\~\\~\\
      
      \textbf{Operation Span Task}:\\
      {\small 간단한 \textbf{산술 문제를 풀면서(정보 조작)} 동시에 \textbf{단어를 기억}해야하는(정보 유지) 과제}\\~\\~\\
      
      \textbf{Mental Rotation Task}:\\
      \vspace{-0.5em}
      \begin{itemize}
        \small
        \item \textbf{학습 단계}: 여러 물체가 배열된 모습 제시
        \item \textbf{지연 단계}: 머릿속으로 그 배열을 90도 회전시키라고 지시
        \item \textbf{검사 단계}: 회전된 모습의 배열을 보여주고 원래 배열을 제대로 회전시킨 것과 일치하는지 판단하게 함
      \end{itemize}
    \end{column}
  \end{columns}
\end{frame}



\section{발표를 맺으며}
\subsection{발표서 다루지 아니한 부분}
\begin{frame}{발표에서 다루지 않았지만 공부한 것들}
  \textbf{교재}:
  \begin{itemize}
    \item 기술들
    \item 여러 세부적인 기억 유형들
    \item 주의력, 심상, 언어, 감정 등 다른 인지과정과 기억의 관계
    \item 기억 장애(예: 알츠하이머병, 외상 후 스트레스 장애)
    \item 동물들의 기억
  \end{itemize}\\~\\

  \textbf{자신이 관심있는 것들 조사 및 정리 후 공유}:
  \begin{itemize}
    \item Multimodal Research
    \item Engram
    \item AI에서의 기억 체계
  \end{itemize}
\end{frame}

\subsection{소감}
\begin{frame}{소감}
  \begin{itemize}
    \large
    \item 단순히 학습 자료에서 얘기하는 바를 그대로 받아들이기보다는 `왜 그럴까'하고 더 생각해보았고, 조원들과 3시간이 부족할 만큼 열정적으로 의견을 주고 받았다. 
      이를 통해 조금 더 깊은 생각과 확장된 시야를 얻을 수 있었다.
    \item 스터디를 완주했다고 해도 기억에 대해 완벽히 이해한 것도 아니며, 오히려 아직도 이해가 안 된 부분과 궁금한 점이 많다. 하지만 본 스터디를 통해 기억에 대해 조금 더 다가갈 수 있었고, 기억을 앞으로 더 심화적으로 공부할 수 있게 해줄 밑거름을 쌓은 것 같다. 기억에 관하여 몰랐던 것들을 많이 알게 되어 정말 유익했던 스터디였다.
  \end{itemize}
\end{frame}


















\end{document}

