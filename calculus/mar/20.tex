\documentclass[../note.tex]{subfiles}

\begin{document}

\section{연쇄법칙}

\subsection{라이프니츠 표현을 사용한 음함수의 미분법}
\begin{example}
  $y^2$을 $x$에 대하여 미분하자, $y$를 미분가능한 $x$의 함수라고 가정하면:
  \begin{align}
    &V = y^2 \\
    &y = f(x) \\
    &V = (f(x))^2 \\
    &\frac{dV}{dx} = \frac{dV}{dy} \cdot \frac{dy}{dx} = 2y \cdot \frac{dy}{dx}
  \end{align}
\end{example}

\begin{note}
  $y$를 미분가능한 $x$의 함수로 가정한다? 사실 존재성을 확인한 후 미분을 시작해야 한다고 한다.
\end{note}

\subsection{함수를 나누어서 풀기}
\begin{example}
  $x^2+y^2=25$를 미분하자:
  $y=\pm\sqrt{25-x^2}$으로 2개의 함수로 나누어 생각할 수 있다.
\end{example}
참고로 $x=-5, x=5$는 domain에 속하지 아니한다. 미분값이 각각 $\mp \infty$로 가기 때문이다. 본 example에서는 2개의 함수로 나누었지만, 실제로는 domain이 다르면 다른 함수이기 때문에 무한히 많은 함수로 나눌 수 있다. 편의상 domain이 가장 큰 함수를 고려하자.

라이프니츠 표현을 추구하자... example로 등장하는 데카르트 엽선 등 그래프가 복잡해지면 함수를 여러개로 나누기란 쉽지 아니하다. 또한 경우의 수를 나누는 것도 쉽지 아니하다.

\begin{theorem}
  $r$은 유리수이다. 다음이 성립한다:
  \begin{equation}
    \frac{d}{dx}(x^r) = rx^{r-1}
  \end{equation}
\end{theorem}
가능한 경우만을 생각한다: domain에 속한 부분만을 고려하면 된다.

\begin{proof}
  핵심 아이디어: 음함수 미분법

  유리수의 정의에 따라 유리수 $r=\frac{m}{n} (m,n \in \ZZ, n \neq 0)$라 하자. $y=x^{\frac{m}{n}} \Rightarrow y^n=x^m$이다. 이를 $x$에 대하여 음함수 미분을 하면 다음과 같다:
  \begin{align}
    ny^{n-1}\frac{dy}{dx} &= mx^{m-1} \\
    \frac{dy}{dx} &= \frac{mx^{m-1}}{ny^{n-1}} = \frac{mx^{m-1}}{nx^{\frac{m}{n}-1}} = \frac{m}{n}x^{\frac{m}{n}-1} \\
                  &= rx^{r-1}
  \end{align}
\end{proof}

\section{로그함수와 역삼각함수의 도함수}
전반부: 로그함수, 후반부: 역삼각함수에 대하여 학습한다.

\begin{definition}{일대일함수}
  임의의 $x_1, x_2 \in X$에 대하여
  \begin{equation}
    x_1 \neq x_2 \rightarrow f(x_1) \neq f(x_2)
  \end{equation}
  이면 함수 $f: X \rightarrow Y$를 일대일 함수\textsuperscript{one to one function, injection function}라고 부른다.
\end{definition}
수평선 판정법을 활용해 판단할 수 있다. ToDo.

\begin{definition}{역함수}
  함수 $f: X\rightarrow Y$가 일대일 함수이면 역함수\textsupersecript{inverse function} $f^{-1}: Y\rightarrow X$로 정의한다.
  \begin{equation}
    f^{-1}(y) = x \leftrightarrow f(x) = y
  \end{equation}
\end{definition}

\begin{remark}
  \begin{itemize}
    \item
      domain과 range가 서로 바뀜.
    \item
      뉘양쓰에 따라 종속변수를 변환함.
      \begin{equation}
        f^{-1}(x)=y \leftrightarrow f(y)=x
      \end{equation}
    \item
      \begin{align}
        f^{-1}(f(x)) = x \\
        f(f^{-1}(x)) = x
      \end{align}
    \item
      $y=x$에 대칭꼴.
  \end{itemize}
\end{remark}

\begin{remark}
  일대일 함수의 특성
  \begin{enumerate}
    \item
      $f$가 일대일이고  연속이면 역함수도 연속이다.
    \item
      $f$가 일대일이고 미분가능하면 역함수는 `수직접선'을 갖는 점을 제외하고 모든점에서 미분가능하다. 
  \end{enumerate}
\end{remark}
2번째에 더하면, 함수의 미분값이 0일때 역함수의 미분값의 절댓값은 $\infity$ 즉 수직선을 가지게 된다. 따라선 역함수가 수직접선을 가지면 그 점에서 미분 불가하다.

\begin{theorem}
  \begin{align}
    \frac{d}{dx}(\log_a{x})=\frac{1}{x\ln{a}} \\
    \frac{d}{dx}(\ln{x})=\frac{1}{x}
  \end{align}
\end{theorem}

\begin{proof}
  핵심 아이디어: 음함수 미분법 사용.

  $y=\log_a{x}$라 하자. $a^y=x$이다. 이를 $x$에 대하여 음함수 미분을 하면 다음과 같다:
  \begin{align}
    a^y\ln{a}\frac{dy}{dx} &= 1 \\
    \frac{dy}{dx} &= \frac{1}{a^y\ln{a}} = \frac{1}{x\ln{a}}
  \end{align}
\end{proof}

\end{doucement}
