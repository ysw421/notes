\documentclass[../note.tex]{subfiles}

\begin{document}

\chapter{Stack}

\begin{definition}[스택]
  쌓아놓은 더미를 스택\textsuperscript{stack}이라고 한다.
\end{definition}
책이 쌓여있는 것을 생각하자. 제일 아래에 깔려있는 책을 ㄱ보기 위해서는 위에 있는 모든 책을 걷어내야 한다.

\subsection{LIFO}
이러한 스택의 특징을 후입선출\textsuperscript{LIFO, Last In First Out}이라고 한다. 즉, 가장 마지막에 들어간 데이터가 가장 먼저 나온다.

\subsection{스택의 구조}
스택의 데이터는 요소\textsuperscript{element}라고 부른다. 가장 위, 아래에 있는 요소는 각각 스택의 상단(top), 하단(bottom)이라고 부른다.

\begin{example}
  재귀함수 또는 함수 안에서 호출되는 함수는 스택의 예시이다. 함수 안에서 새로운 함수가 호출되면 호출된 함수가 끝나기까지 다른 명령어가 실행되지 아니한다. 나중에 호출된 함수가 가장 먼저 실행되는 것이 스택과 같다.
\end{example}

\section{ADT}
스택의 간단한 ADT에 대하여 알아본다. 스택 객체는 0개 이상의 원소를 가지는 유한 선형 리스트이다. 유한 선형 리스트이기에 --- 동적으로 스택 최대 크기를 끊임없이 늘려준다고 한들, 최대 크기가 존재하나 해당 크기를 늘려주는 것이니 유한하다고 표현한다. --- 스택은 최대 크기를 가진다. 스택은 기본적으로 자료를 저장하는 도구이기 때문에 자료를 추가하고 조회하는 연산을 가진다. 주요 연산은 다음과 같다:
\begin{itemize}
  \item \texttt{create(size)}: 최대 크기가 size인 스택을 생성한다.
  \item \texttt{isEmpty(s)}: 스택이 비어있는지 확인한다.
  \item \texttt{isFull(s)}: 스택이 가득 차있는지 확인한다.
  \item \texttt{push(s, item)}: 스택에 원소를 추가한다.
  \item \texttt{pop(s)}: 스택에서 원소를 제거하고 제거한 값을 반환한다.
  \item \texttt{peek(s)}: 스택의 상단에 있는 원소를 반환한다.(제거하지 않음)
\end{itemize}
push, pop, peek 모두 명령을 실행하기 사전에 스택이 가득 차있는지 고려해야 한다.

\begin{verbatim}
#include <stdio.h>
#include <stdlib.h>
#include <stdbool.h>

#define MAX_STACK_SIZE 100
typedef int element;
element stack[MAX_STACK_SIZE];
int top = -1;

bool is_empty() {
  return (top == -1);
}

bool is_full() {
  return (top == (MAX_STACK_SIZE - 1));
}

void push(element item) {
  if (is_full()) {
    fprintf(stderr, "스택 포화 에러\n");
    return;
  }
  else stack[++top] = item;
}

element pop() {
  if (is_empty()) {
    fprintf(stderr, "스택 공백 에러\n");
    exit(1);
  }
  else return stack[top--];
}

int main() {
  push(1);
  push(2);
  push(3);
  printf("%d\n", pop());
  printf("%d\n", pop());
  printf("%d\n", pop());
  return 0;
}
\end{verbatim}
비단, 이와같이 스택을 구현하게 되면 스택은 변수에 종속적이다. 각 함수가 하나의 스택만을 위한 것이기 때문에 좋은 코드가 아니다.
% \lstinputlisting[language=c, caption=NEURON module, label=lst:neuron_example]{./code/mar-25-1.c}

ToDo...
구조체 배열을 활용한 스택, 동적 스택

\end{document}
